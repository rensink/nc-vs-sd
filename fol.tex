\section{Equivalence to First-Order Logic}

Nested conditions can express precisely the same properties of graphs as First-Order Logic (FOL). This in fact follows from previous work, but since later on in this paper we strengthen the result to the existence of functors between the respective categories, we start out by recalling the relevant definitions and giving a direct proof of the equivalence. This will also help in developing an intuition about nested conditions.

We restrict to binary predicates $\la,\lb$, taken from a set $\Lab$; furthermore, we use variables $x,y$ taken from a set $\Var$. The syntax of FOL that we use is given by the grammar
%
\[ \phi \:::=\: \True
        \:\mid\: \False
		\:\mid\: \la(x_1,x_2)
        \:\mid\: x_1=x_2
		\:\mid\: \phi_1\wedge \phi_2
		\:\mid\: \phi_1\vee \phi_2
		\:\mid\: \neg\phi_1
		\:\mid\: \exists X\st \phi_1 
		\]
where $\la\in\Lab$ is an arbitrary predicate name, $x_1,x_2\in\Var$ are arbitrary variables and $X\subseteq \Var$ an arbitrary set of variables. We use the common $\exists x\st \phi$ as syntactic sugar for $\exists\setof x\st \phi$. We also use the concept of \emph{free variables} of $\phi$, denoted $\fv(\phi)$, inductively defined in the usual way.

For the correspondence of FOL to nested conditions, we equate $\Var$ to the universe of node identities. An open graph $g\of I\to G$ can then be seen as an assignment of free variables (the elements of $I$) to a given domain (the graph $G$) in which the existence of a $\la$-labelled edge between $g(x)$ and $g(y)$ means that the predicate $\la(x,y)$ holds. Formally, satisfaction is captured by a relation $\sat$ inductively defined over the structure of the formula. In fact, for reasons to become clear below, we defined $g\sat \phi$ for arbitrary $g\of A\to B$ such that $\fv(\phi)\subseteq N_A$, as follows:
%
\[\begin{array}{ll}
g\sat \True & \text{always} \\
g\sat \False & \text{never} \\
g\sat \la(x_1,x_2) & \iffdef \text{there is an } e\in E_G \text{ with } s(e)=g(x_1), \ell(e)=\la, t(e)=g(x_2) \\
g\sat x_1=x_2 & \iffdef g(x_1)=g(x_2) \\
g\sat \phi_1\wedge \phi_2 & \iffdef g\sat \phi_1 \text{ and } g\sat \phi_2 \\
g\sat \phi_1\vee \phi_2 & \iffdef g\sat \phi_1 \text{ or } g\sat \phi_2 \\
g\sat \neg\phi_1 & \iffdef \text{not } g\sat \phi_1 \\
g\sat \exists X\st \phi_1  & \iffdef h\sat \phi_1 \text{ for some } h \text{ with } h\setminus X =g\setminus X \enspace.
\end{array}\]
%
When comparing the satisfaction relations for nested conditions and FOL formulas, a technical issue is that the sets of models for which they are defined are not the same: as seen above, $g\sat\phi$ is defined for $g\of A\to G$ whenever $\fv(\phi)\subseteq N_A$, whereas $g\sat\cT$ for a nested condition $\cT$ is only defined if $A$ is \emph{precisely} $I_\cT$. Due to this discrepancy, there cannot be any $\phi$ and $\cT$ such that $g\sat \phi$ if and only if $g\sat \cT$; instead, the best we can hope for is that this holds whenever $A=I_{\fv(\phi)}$. The following property implies that this is indeed all we need to be interested in, since $g$-images for $A$-elements outside $\fv(\phi)$ do not make a difference for satisfaction.

\begin{proposition}\label{prop:free vars only}
Let $\phi$ be a FOL-formula and $g\of A\to G$ such that $\fv(\phi)\subseteq N_A$; then $g\sat \phi$ if and only if $g\restr I_{\fv(\phi)}\sat \phi$.
\end{proposition}

$g\sat \phi$ if and only if $g\restr I_\cT \sat \cT$. However, since the $g$-images outside $\fv(\phi)$ do not matter for $g\sat \phi$ (in formal terms, $g_1\restr \fv(\phi)=g_2\restr \fv(\phi)$ implies $g_1\sat\phi \Longleftrightarrow g_2\sat\phi$), we still think of such $\phi$ and $\cT$ as ``satisfying the same models'' and therefore being essentially equivalent.

\begin{definition}[essential equivalence]\label{def:essential equivalence}
A formula $\phi$ and a condition tree $\cT$ are \emph{essentially equivalent} if for all $g\of I\to G$, $g\sat \phi$ if and only if $g\restr I_\cT\sat \cT$.
\end{definition}

In the remainder of this section, we show that
%
\begin{inumerate}
\item for every nested condition, there is an essentially equivalent FOL formula, and
\item for every FOL formula, there is an essentially equivalent nested condition.
\end{inumerate}
%
\subsection{From nested conditions to FOL}

We first define formulas for graphs, and then (inductively) for nested conditions. Let $P$ be an arbitrary graph.
%
\[ \phi_P = \textstyle
% \bigwedge_{x\in N_P}{x=x} \wedge
 \bigwedge_{e\in E_P} \ell(e)\bigl(s(e),t(e)\bigr)
\]
%
The essential property of $\phi_P$ is the following.
%
\begin{proposition}\label{prop:graph formula}
Let $P$ be an arbitrary graph.
\begin{enumerate}
\item If $g\of P\to G$ is an arbitrary graph morphism, then $g\sat \phi_P$;
\item If $g\of A\to G$ such that $g\sat \phi_P$, then $k\sat \phi_P$ for some $k\of P\to G$.\todo{This is probably not yet quite what we need}
\end{enumerate}
\end{proposition}
%
In the definition below, we assume (without loss of generality, because we can always rename node identities up to isomorphism) that for all nodes $v$ of a nested condition, if $u_v\of I_v\to Q_v$ then $Q_v\cap P_v=\emptyset$; in other words, on each successive layer of the condition, the node identities (here treated as variables) are distinct. 

Let $\cT$ be an arbitrary nested condition. We define $\phi_v$ for all nodes $v\in V_\cT$, inductively on the depth of the subtree $\hat v$:
%
\[ \phi_v = \textstyle \exists N_{P_v}\st \phi_{P_v}\wedge \bigwedge_{x\in N_{I_v}} d_v(x)=u_v(x) \wedge \bigwedge_{w\childof v} \neg\phi_w \enspace. \]
%
Finally, we define $\phi_\cT=\phi_{\rt\cT}$.
%
\begin{theorem}
For an arbitrary nested tree $\cT$ and an arbitrary open graph $g\of I_\cT\to G$, $g\sat\phi_\cT$ if and only if $g\sat\cT$.
\end{theorem}
%
\begin{proof}
For our proof, we need an induction hypothesis for all nodes $v$ of a nested condition, in which the $u_v$-morphism is also accounted for. The property we actually prove is therefore:
\begin{quote}
For all nodes $v$ of $\cT$ with $u_v\of I_v\to Q_v$ and all $g\of Q_v\to G$, $g\sat \phi_v$ if and only if $u_v;g\sat \hat v$.
\end{quote}
This is proved inductively: we assume that the property holds for all $w\childof v$, and show that it then holds for $v$. Since $u_{\rt\cT}=\id_{I_\cT}$ and hence $Q_{\rt\cT}=I_\cT$, this implies the proof obligation.
%
\begin{description}
\item[If.] Assume $u_v;g\sat \hat v$, meaning that there is some $h\of P_v\to G$ such that $d_v;h=u_v;g$ and for all $w\childof v$, $u_w;h\nsat \hat w$. Since (by assumption) the node sets of $P_v$ and $Q_v$ are disjoint, we can construct the graph $A=P_v\cup Q_v$ and the graph morphism $k=g\cup h\of A\to G$. Note that $P_v=A\setminus N_{P_v}$ and hence $g=k\setminus N_{P_v}$.


By \Cref{prop:graph formula}, it follows that $h\sat P_v$. Moreover, by the induction hypothesis, $h\nsat \phi_w$, implying $h\sat\neg \phi_w$. \Cref{prop:free vars only} then implies $k\sat P_v$ and $k\sat \neg\phi_w$. Also, for any $x\in N_{I_v}$, $k(d_v(x))=g(d_v(x))=h(u_v(x))=k(u_v(x))$, hence $k\sat d_v(x)=u_v(x)$. All in all, we have
%
\begin{equation}\label{eq:k}
k\sat \textstyle \phi_{P_v}\wedge \bigwedge_{x\in N_{I_v}} d_v(x)=u_v(x) \wedge \bigwedge_{w\childof v} \neg\phi_w \enspace.
\end{equation}
%
Since $g\setminus N_{P_v}=g=k\setminus N_{P_v}$, it follows that $g\sat \phi_v$.

\item[Only if.] Assume $g\sat \phi_v$, and let $k$ be such that $g\setminus N_{P_v}=k\setminus N_{P_v}$ and $k$ satisfies \Cref{eq:k}. Note that in fact $g\setminus N_{P_v}=g$ because $P_v$ and $Q_v$ are  graphs on disjoint nodes. Let $h=k\setminus N_{Q_v}$; then \todo{to be completed}
\end{description}
\end{proof}
%
The formula $\phi_\cT$ can in practice be simplified by alpha-converting the variable names such that they are shared as much as possible, rather than being distinct on each layer as required for the definition; depending on the condition tree, this may allow many or even all of the equations $d_v(x)=u_v(x)$ in $\phi_v$ to be omitted. This is in particular the case if both $d_v$ and $u_v$ are injective.

\subsection{From FOL to nested conditions}

This is shown inductively over the structure of FOL formulas. Indeed, we show that
\begin{itemize}
\item for every basic predicate in FOL, there exists a corresponding nested condition;
\item for every operator in FOL there exists a construction over nested conditions that mimics it.
\end{itemize}
%
Because of the De Morgan duality $\phi\vee \phi\equiv \neg(\neg\phi\wedge \neg\psi)$, in fact we only have to give constructions (of the second kind) for $\wedge$, $\neg$ and $\exists$.

