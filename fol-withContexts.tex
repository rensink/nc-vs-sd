\section{Equivalence to First-Order Logic}

Nested conditions can express precisely the same properties of graphs as First-Order Logic (FOL). This in fact follows from previous work, but since later on in this paper we strengthen the result to the existence of functors between the respective categories, we start out by recalling the relevant definitions and giving a direct proof of the equivalence. This will also help in developing an intuition about nested conditions.

We restrict to binary predicates $\la,\lb$, taken from a set $\Lab$; furthermore, we use variables $x,y$ taken from a set $\Var$. The syntax of FOL that we use is given by the grammar
%
\[ \phi \:::=\: \True
        \:\mid\: \False
		\:\mid\: \la(x_1,x_2)
        \:\mid\: x_1\Eq x_2
		\:\mid\: \phi_1\wedge \phi_2
		\:\mid\: \phi_1\vee \phi_2
		\:\mid\: \neg\phi
		\:\mid\: \exists X\st \phi 
		\]
where $\la\in\Lab$ is an arbitrary predicate name, $x_1,x_2\in\Var$ are arbitrary variables and $X\subseteq \Var$ an arbitrary set of variables. We use the common $\exists x\st \phi$ as syntactic sugar for $\exists\setof x\st \phi$. We also use the concepts of \emph{free variables} and \emph{bound variables} of $\phi$, denoted $\fv(\phi)$ and $\bv{\phi}$, respectively, both inductively defined in the usual way.\todo{AC: is $x$ bound in $\exists x \st \True$? Both answers are reasonable, I think. Here we assume it is.} Note that unlike  $\fv(\_)$, $\bv{\_}$ is not invariant under $\alpha$-conversion.


As common practice in categorical logic \cite{JacobsBook}, we consider \emph{formulas in context} of the kind $\ctx{X}{\phi}$, where the context $X$ is a set of variables containing $\fv(\phi)$. As an additional constraint, we require that $X$ is disjoint from $\bv(\phi)$: This will simplify the correspondence between formulas and nested conditions without loss of generality, because bound variables can be renamed if needed.

\begin{definition}[formulas in context]
	\label{def:formula in context}
	Let $X \subseteq \Var$ and $\phi$ be a FOL formula. Then $\ctx{X}{\phi}$ is a \emph{formula in context} if it can be obtained using the following axioms and formation rules:
	
	$$\infer{\ctx{\emptyset}{\True}}{} \qquad 
	\infer{\ctx{\emptyset}{\False}}{} \qquad 
	\infer{\ctx{\{x_1, x_2\}}{\la(x_1,x_2)}}{} \qquad 
	\infer{\ctx{\{x_1, x_2\}}{x_1\Eq x_2}}{}$$
	$$\infer[{[\mbox{ctx-}\wedge]}]{\ctx{X}{\phi_1 \wedge \phi_2}}{\ctx{X}{\phi_1} & \ctx{X}{\phi_2}} \quad 
	\infer[{[\mbox{ctx-}\vee]}]{\ctx{X}{\phi_1 \vee \phi_2}}{\ctx{X}{\phi_1} & \ctx{X}{\phi_2}} \quad 
	\infer[{[\mbox{ctx-}\neg]}]{\ctx{X}{\neg \phi}}{\ctx{X}{\phi}}$$
	$$\infer[{[\mbox{ctx-}\exists]}]{\ctx{Z \setminus X}{\exists X \st\phi}}{\ctx{Z}{\phi}} \quad 
\infer[{[\mbox{ctx-weak}]}]{\ctx{Y}{\phi}}{\ctx{X}{\phi} & X\subseteq Y & \bv(\phi)\cap Y = \emptyset}$$
\end{definition}
%
For a formula in context $\ctx{X}{\phi}$ it is easy to verify that $\fv(\phi) \subseteq X$ and, thanks to the condition in the weakening rule, that $\bv(\phi) \cap X = \emptyset$. This implies that $\phi$ (and all its sub-formulas) cannot contain a free variable which is bound in a sub-formula. 

\medskip

In this section we use the following concrete category of graphs.
%
\begin{definition}
An edge-labelled multi-graph is a tuple $\tupof{N,E,s,t,\ell}$ where
\begin{itemize}
\item $N$ is a set of nodes;
\item $E$ is a set of edges;
\item $s,t\of E\to N$ are the source and target function, respectively;
\item $\ell\of E\to \Lab$ is the edge labelling function.
\end{itemize}
A multi-graph morphism $f\of A\to B$ is a tuple $f=\tupof{f\N,f\E}$ where $f\N\of N_A\to N_B$ and $f\E\of E_A\to E_B$ are functions satisfying $f\E;s_B=s_A;f\N$, $f\E;t_B=t_A;f\N$ and $f\E;\ell_B=\ell_A$.
\end{definition}
%
Although our theory mostly abstracts from node and edge identities by relating graphs through morphisms, in the context of FOL interpretation we sometimes have to rely on meaningful, fixed node identities (namely where they correspond to variables). We call $A$ and $B$ \emph{consistent} if their source, target and label functions coincide on $E_A\cap E_B$, and \emph{disjoint} if $N_A\cap N_B=E_A\cap E_B=\emptyset$ (in which case they are certainly consistent).  If $A$ and $B$ are disjoint, we use $A\cup B$ to denote their union; also, if $g\of A\to G$ and $h\of B\to G$ for disjoint $A$ and $B$, then $g\cup h\of (A\cup B)\to G$ denotes the union of $g$ and $h$. We call $A$ a subgraph of $B$, denoted $A\subseteq B$, if $B=A\cup B$; likewise, $f$ is a sub-morphism of $g$, denoted $f\subseteq g$, if $f\cup g=g$. If $A$ and $B$ are consistent, then $B\setminus A$ denotes the largest subgraph of $B$ with node set $N_B\setminus N_A$, and for $g\of A\to G$ we use $g\restr B$ and $g\setminus B$ to denote the restriction of $g$ to (the nodes and edges of) $A \cap B$ and $A\setminus B$, respectively.\todo{We use $g\setminus B$ also if $B$ is not a subgraph of $A$, e.g., in the definition of $g\sat \exists E\st \phi_1$. I think it is actually well-defined.} Finally, if $X$ is a set then $\tupof X$ will denote the discrete (i.e., edge-less) graph with node set $X$.

For the correspondence of FOL to nested conditions, we equate $\Var$ to the universe of node identities. A graph morphism $g\of A\to G$ can then be seen as an assignment of variables (the nodes of $A$, playing the role of the context) to a given domain (the graph $G$) in which the existence of an $\la$-labelled edge between $g(x)$ and $g(y)$ means that the predicate $\la(x,y)$ holds. Formally, satisfaction is captured by a relation $\sat$ between morphisms and formulas in context: we write $g \sat \phi$ as a shorthand for $g \of A \to G \sat \ctx{N_A}{\phi}$. As a consequence, asserting $g \sat \phi$ has an implicit proof obligation, i.e. that $\ctx{N_A}{\phi}$ is a formula in context.  Satisfaction is defined inductively over the structure of the formula.
% for a morphism $g\of A \to G$, by $g \sat \phi$ we mean that $g$ satisfies the formula in context $\ctx{N_A}{\phi}$. 
For reasons to become clear below, the source graph $A$ of $g$ is not required to be discrete.
%   In fact, for reasons to become clear below, we defined $g\sat \phi$ for arbitrary $g\of A\to G$ such that $\fv(\phi)\subseteq N_A$, without requiring that $A$ is discrete.
%
\[\begin{array}{ll}
g\sat \True & \text{always} \\
g\sat \False & \text{never} \\
g\sat \la(x_1,x_2) & \iffdef \text{there is an } e\in E_G \text{ with } s(e)=g(x_1), \ell(e)=\la, t(e)=g(x_2) \\
g\sat x_1\Eq x_2 & \iffdef g(x_1)=g(x_2) \\
g\sat \phi_1\wedge \phi_2 & \iffdef g\sat \phi_1 \text{ and } g\sat \phi_2 \\
g\sat \phi_1\vee \phi_2 & \iffdef g\sat \phi_1 \text{ or } g\sat \phi_2 \\
g\sat \neg\phi & \iffdef \text{not } g\sat \phi \\
g\sat \exists X\st \phi  & \iffdef h\sat \phi \text{ for some $h\of B \to G$ such that $A \subseteq B$,}\\
& \text{ $X \subseteq N_B$, and }  h\restr A = g\enspace.
\end{array}\]
%
Note that the last clause requires that $\ctx{N_A}{\exists X\st \phi}$ is a formula in context, which implies that $X \cap \bv(\phi) = \emptyset$.\todo{AC: comment more on this?}

\bigskip

\hrule

\medskip
\todo[inline]{AC: this can be deleted, I think.}

When comparing the satisfaction relations for nested conditions and FOL formulas, a technical issue is that the sets of models for which they are defined are not the same: as seen above, $g\sat\phi$ is defined for $g\of A\to G$ whenever $\fv(\phi)\subseteq N_A$, whereas $g\sat\cT$ for a nested condition $\cT$ is only defined if $A$ is \emph{precisely} $I_\cT$. Due to this discrepancy, there cannot be any $\phi$ and $\cT$ such that $g\sat \phi$ if and only if $g\sat \cT$; instead, the best we can hope for is that this holds whenever $A=\tupof{\fv(\phi)}$. The following property implies that this is indeed all we need to be interested in, since $g$-images for $A$-elements outside $\fv(\phi)$ do not make a difference for satisfaction.

\begin{proposition}\label{prop:free vars only-old}
Let $\phi$ be a FOL-formula and $g\of A\to G$ such that $\fv(\phi)\subseteq N_A$; then $g\sat \phi$ if and only if $g\restr \tupof{\fv(\phi)}\sat \phi$.
\end{proposition}
%
Because of this property, when constructing a morphism $g\of A\to G$ such that $g\sat \phi$, we may assume $A$ to be any graph for which $\tupof{\fv(\phi)}\subseteq A$, without loss of generality.

\medskip

\hrule

The following result states that to check if a morphism $g:A \to G$ satisfies a formula $\phi$, it is sufficient to consider the action of the morphism on the free variables of $\phi$.

\todo[inline]{AC: the condion is stronger, using formulas in context.}
\begin{proposition}\label{prop:free vars only}
	Let $\phi$ be a FOL-formula and $g\of A\to G$ such that $\ctx{N_A}{\phi}$ is a formula in context; then $g\sat \phi$ if and only if $g\restr \tupof{\fv(\phi)}\sat \phi$.
\end{proposition}
%
Because of this property, when constructing a morphism $g\of A\to G$ such that $g\sat \phi$, we may assume $A$ to be any graph for which $\tupof{\fv(\phi)}\subseteq A$ and such that $\ctx{N_A}{\phi}$ is a formula in context.

\medskip

In the remainder of this section, we show that
%
\begin{inumerate}
\item for every nested condition, there is an 
%essentially 
equivalent FOL formula, and
\item for every FOL formula, there is an 
% essentially 
equivalent nested condition.
\end{inumerate}
%
\subsection{From nested conditions to FOL}

We first define formulas in context for graphs, and then (inductively) for nested conditions. Given a graph $P$, we define
\begin{align}
	\phi_P = \textstyle
% \bigwedge_{x\in N_P}{x=x} \wedge
 \bigwedge_{e\in E_P} \ell(e)\bigl(s(e),t(e)\bigr)\label{eq:phiP}
\end{align}
%
Note that $\fv(\phi_P)\subseteq N_P$ and that $\ctx{N_P}{\phi_P}$ is a formula in context. The essential property of $\ctx{N_P}{\phi_P}$ is the following.
%
\begin{proposition}\label{prop:graph formula}
Let $P$ be a graph.
\begin{enumerate}
\item If $g\of P\to G$ is a graph morphism, then $g\sat \phi_P$;
\item If $g\of \tupof{N_P}\to G$ and $g\sat \phi_P$, then there is a $g'\of P\to G$ such that $g\subseteq g'$ and $g' \sat \phi_P$.\todo{AC: This item is subsumed by the next one, which is used directly in the proof of Theorem\ref{th:comparing satisfactions}. Delete it after checking where it is used.}
\item If $g:A\to G$ is a graph morphism, $\tupof{N_P} \subseteq A$ and $g\sat \phi_P$, then there is a $g': A' \to G$ such that $P \subseteq A'$, $A \subseteq A'$, $N_{A'} = N_A$, $g' \restr A = g$ and $g' \sat \phi_P$. 
\end{enumerate}
\end{proposition}
%

To lift this the definition from graphs to nested conditions, it is convenient to assume that some of the graphs appearing in the latter are disjoint. We can do this without loss of generality, because we can always rename node and edge identities up to isomorphism.

\begin{definition}
	A nested condition $\cT$ is \emph{\proper} if for all pair of nodes $v, w$ of $\cT$, if $w$ is a descendant of $v$ (i.e.\ $v {\ancestof} w$), then $P_v$ and $P_w$ 
	%$N_{P_v}$ and $N_{P_w}$ 
	are disjoint. 
\end{definition}

% [Old version] In the definition below, we assume (without loss of generality, because we can always rename node identities up to isomorphism) that for all nodes $v$ of a nested condition, $U_v$ and $P_v$ are disjoint; in other words, on each successive layer of the condition, the node identities (here treated as variables) are distinct. 

Let $\cT$ be an {\proper} nested condition. We define $\psi_v,\phi_v$ for all nodes $v\in V_\cT$, inductively on the depth of the subtree $\hat v$:
%
\begin{align}
\psi_v = {} & \textstyle \phi_{P_v}\wedge \bigwedge_{x\in N_{I_v}} d_v(x)\Eq u_v(x) \wedge \bigwedge_{w\childof v} \neg\phi_w \label{eq:psi} \\
\phi_v = {} & \exists N_{P_v}\st \psi_v\label{eq:phi} 
\end{align}
%
Finally, we define $\phi_\cT=\phi_{\rtof\cT}$.
%

\todo[inline]{AC: Add examples? eg for some limit cases like true and false?}

\begin{proposition}
	\label{prop:formula in context}
	Let $\cT$ be a {\proper} nested condition, $v \in V_\cT$ be a node and $\psi_v, \phi_v$ be the FOL formula defined in \eqref{eq:psi} and \eqref{eq:phi}. Then $\ctx{N_{U_v}\cup N_{P_v}}{\psi_v}$ and $\ctx{N_{U_v}}{\phi_v}$ are formulas in context. 
 
	As a consequence, 	 $\ctx{N_{I_\cT}} {\phi_\cT}$ is a formula in context. 
\end{proposition}
\begin{proof}
	% We prove by induction that $\ctx{N_{U_v}}{\phi_v}$ is a formula in context for each node $v \in V_\cT$. 
	The second statement immediately follows from the first one, since  $\phi_\cT = \phi_{\rtof\cT}$ and $U_{\rtof\cT} = I_{\rtof\cT} = I_\cT$.\\
%
For the first statement, we proceed by induction, assuming that the property holds for all $w\childof v$ and showing that it then holds for $v$.
	Let $v\in V_\cT$, and assume by inductive hypothesis that $\ctx{N_{U_w}}{\phi_w}$ is a formula in context for all $w\childof v$. 
	
	To show that $\ctx{N_{U_v} \cup N_{P_v}}{\psi_v}$ is a formula in context, by rule [ctx-$\wedge$] it is sufficient to show that 
	(a) $\ctx{N_{U_v} \cup N_{P_v}}{\phi_{P_v}}$, (b) $\ctx{N_{U_v} \cup N_{P_v}}{\bigwedge_{x\in N_{I_v}} d_v(x)\Eq u_v(x)}$, and (c) $\ctx{N_{U_v} \cup N_{P_v}}{\bigwedge_{w\childof v} \neg\phi_w}$ are formulas in context. For (a), by the observation after \eqref{eq:phiP} we know that $\ctx{N_{P_v}}{\phi_{P_v}}$ is a formula in context, and we can conclude using rule [ctx-weak]: in fact $\bv(\phi_{P_v}) = \emptyset$.
	%
	For (b), note that for all $x \in N_{I_v}$ we have that $\ctx{\{d_v(x),u_v(x)\}}{d_v(x)\Eq u_v(x)}$ is a formula in context, that $\{d_v(x),u_v(x)\} \subseteq N_{U_v} \cup N_{P_v}$, and $\bv(d_v(x)\Eq u_v(x))= \emptyset$. Therefore using rules [ctx-weak] and [ctx-$\wedge$] we obtain that $\ctx{N_{U_v} \cup N_{P_v}}{\bigwedge_{x\in N_{I_v}} d_v(x)\Eq u_v(x)}$ is a formula in context. Finally, for (c), for all $w\childof v$ first observe that $U_w = P_v$. Thus by inductive hypothesis we can assume that $\ctx{N_{P_v}}{\phi_w}$ is a formula in context. Furthermore,  since $\cT$ is {\proper}, no node of $U_v$ can be bound in $\phi_w$, thus applying rules [ctx-$\neg$] and [ctx-weak] we obtain that $\ctx{N_{U_v} \cup N_{P_v}}{\neg\phi_w}$ is a formula in context, and so is 
	$\ctx{N_{U_v} \cup N_{P_v}}{\bigwedge_{w\childof v} \neg\phi_w}$ by applying rule [ctx-$\wedge$].

	To show that $\ctx{N_{U_v}}{\phi_v}$ is a formula in context, observe that since $\ctx{N_{U_v} \cup N_{P_v}}{\psi_v}$ is a formula in context and $N_{U_v}$ and $N_{P_v}$ are disjoint (because $\cT$ is {\proper}), we have, as desired:

	$$\infer[{[\mbox{ctx-}\exists]}]
	{\ctx{N_{U_v}}{\phi_v = \exists N_{P_v}\st \psi_v}}
	{\ctx{N_{U_v} \cup N_{P_v}}{\psi_v}}$$
	\qed	
\end{proof}

\begin{theorem}
	\label{th:comparing satisfactions}
	Let $\cT$ be a {\proper} nested condition, $v \in V_\cT$ be a node and $g\of U_v \to G$ be a graph morphism. Then $g\sat \phi_v$ if and only if $u_v;g\sat \hat v$. \todo{AC: use different simbols for the two notions of satisfaction?}
	
	As a consequence, if $g: I_\cT \to G$, then $g\sat\phi_\cT$ if and only if $g\sat\cT$.
\end{theorem}
%
\begin{proof}
	The second statement follows from the first one, because $\phi_\cT = \phi_{\rtof\cT}$ and $U_{\rtof\cT} = I_{\rtof\cT} = I_\cT$.\\	
% For our proof, we need an induction hypothesis for all nodes $v$ of a nested condition, in which the $u_v$-morphism is also accounted for. The property we actually prove is therefore:
%\begin{quote}
%For all nodes $v$ of $\cT$ and all $g\of U_v\to G$, $g\sat \phi_v$ if and only if $u_v;g\sat \hat v$.
%\end{quote}
%
%This is proved inductively: we assume that the property holds for all $w\childof v$, and show that it then holds for $v$. Since $u_{\rtof\cT}=\id_{I_\cT}$ and hence $U_{\rtof\cT}=I_\cT$, this implies the proof obligation.
%
As in the previous proposition we proceed by induction, assuming that the property holds for all $w\childof v$, and showing that it then holds for $v$.
\begin{description}
\item[If.] Assume $u_v;g\sat \hat v$, meaning that there is some $h\of P_v\to G$ such that $d_v;h=u_v;g$ and $u_w;h\nsat \hat w$ for all $w\childof v$. Since $P_v$ and $U_v$ are disjoint (because $\cT$ is {\proper}), we can construct the graph $A=P_v\cup U_v$ and the graph morphism $k=g\cup h\of A\to G$. Note that 
$g = k \restr U_v$: 
%$P_v=A\setminus N_{P_v}$ and hence $g=k\setminus N_{P_v}$.

\smallskip
By \Cref{prop:graph formula}, it follows that $h\sat \phi_{P_v}$. Moreover, by the induction hypothesis, $h\nsat \phi_w$, implying $h\sat\neg \phi_w$. \Cref{prop:free vars only} then implies $k\sat \phi_{P_v}$ and $k\sat \neg\phi_w$, given that both $\ctx{N_{U_v} \cup N_{P_v}}{\phi_v}$ and $\ctx{N_{U_v} \cup N_{P_v}}{\neg \phi_w}$ are formulas in context by \Cref{prop:formula in context}.  Also, for any $x\in N_{I_v}$,
\[ k(d_v(x))=h(d_v(x))=(d_v;h)(x) = (u_v;g)(x) = g(u_v(x))=k(u_v(x)) \enspace, \]
hence $k\sat d_v(x)\Eq u_v(x)$. All in all, we have $k\sat \psi_v$, where $\ctx{N_{U_v} \cup N_{P_v}}{\psi_v}$ is a formula in context by \Cref{prop:formula in context}. Since $g = k \restr U_v$, by the definition of satisfaction we have  $g\sat \exists N_{P_v}\st \psi_v$, i.e.\ $g\sat \phi_v$. 

\smallskip
\item[Only if.] Assume $g\sat \phi_v = \exists N_{P_v}\st \psi_v$. This means that 
%$\ctx{N_{U_v}}{\phi_v}$ is a formula in context, that 
there is a $k\of A\to G$ such that $U_v \subseteq A$, $N_{P_v} \subseteq N_A$, $k\restr U_v = g$ and $k \sat \psi_v$. Since  $k \sat \psi_v$, we know that $k \sat \phi_{P_v}$. Then by \Cref{prop:graph formula} there is a $k':A' \to G$ such that $P_v {\subseteq} A'$, $A\subseteq A'$, $N_{A'} = N_A$, $k'\restr A = k$ and $k' \sat \phi_{P_v}$. Let $h: P_v \to G$ be $P_v \hookrightarrow A' \stackrel{k'}{\to} G$ : we show that $h$ is a witness for $g \sat \hat v$.

First, note that $k'\sat \psi_v = \left( \phi_{P_v}\wedge \bigwedge_{x\in N_{I_v}} d_v(x)\Eq u_v(x) \wedge \bigwedge_{w\childof v} \neg\phi_w \right)$ by \Cref{prop:free vars only}, because $k\sat \psi_v$ and $N_{A'}=N_A$. 
\begin{itemize}

\item $k'\sat \bigwedge_{x\in N_{I_v}} d_v(x)\Eq u_v(x)$ implies that $h(d_v(x))=k'(d_v(x))=k'(u_v(x))= k(u_v(x)) = g(u_v(x))$ for all $x\in N_{I_v}$, and since $I_v=\tupof{N_{I_v}}$ this implies $d_v;h=u_v;g$.

\item $k'\sat \neg\phi_w$ (for all $w\childof v$) implies $h\sat\neg\phi_w$ (by \Cref{prop:free vars only}, since $\ctx{N_{P_v} = N_{U_w}}{\phi_w}$ is a formula in context by \Cref{prop:formula in context}), hence $h\nsat\phi_w$, hence (by the induction hypothesis) $u_w;h'\nsat \hat w$.

\end{itemize}


% Since $P_v$ and $U_v$ are disjoint, due to \Cref{prop:free vars only} we may assume without loss of generality that $A= U_v\cup\tupof{N_{P_v}}$, and hence $k=g\cup h$ for some $h\of \tupof{N_{P_v}}\to G$. Based on this, we can show the existence of a witness for $g\sat \hat v$, as follows.

% \begin{itemize}
% \item $k\sat \phi_{P_v}$ implies (by \Cref{prop:free vars only}) $h\sat \phi_{P_v}$, and hence (by \Cref{prop:graph formula}) $h\subseteq h'$ for some $h'\of P_v\to G$. \Cref{prop:free vars only} then implies $h'\sat \phi_{P_v}$.

% \item Let $k'=g\cup h'$; then $k'\supseteq k$ and hence (again by \Cref{prop:free vars only}) $k'\sat \psi_v$.

% \item $k'\sat \bigwedge_{x\in N_{I_v}} d_v(x)\Eq u_v(x)$ means that $h'(d_v(x))=k'(d_v(x))=k'(u_v(x))=g(u_v(x))$ for all $x\in N_{I_v}$, and since $I_v=\tupof{N_{I_v}}$ this implies $d_v;h'=u_v;g$.

% \item $k'\sat \neg\phi_w$ (for all $w\childof v$) implies (by \Cref{prop:free vars only}) $h'\sat\neg\phi_w$, hence $h'\nsat\phi_w$, hence (by the induction hypothesis) $u_w;h'\nsat \hat w$.
% \end{itemize}

% Taken together, this establishes that $h'$ is a witness for  $g\sat \phi_v$.
\end{description}
\end{proof}
%
The formula $\phi_\cT$ can in practice be simplified by alpha-converting the variable names such that they are shared as much as possible, rather than being distinct on each layer as required for the definition; depending on the condition tree, this may allow many or even all of the equations $d_v(x)\Eq u_v(x)$ in $\phi_v$ to be omitted. This is in particular the case if both $d_v$ and $u_v$ are injective. From the logical pont of view, this can be obtained by applying repeatedly the law $\left( \exists x \st x = y \wedge P(x)\right) \equiv P(y)$.

\subsection{From formulas in context to nested conditions}

This is shown inductively over the structure of FOL formulas in context. Indeed, we show that 
\begin{itemize}
\item for every axiom in \Cref{def:formula in context} of the shape $\ctx{X}{\phi}$ there exists a corresponding {\proper} nested condition $\cT_\phi$ with root interface $\tupof{X}$;
\item for every formation rule in \Cref{def:formula in context} there exists a construction over {\proper} nested conditions that mimics it.
\end{itemize}
%
% The main result then shows that for each formula in context $\ctx{X}{\phi}$ and graph morphism $g: \tupof{X} \to G$, $g \sat \phi$ if and only if $g \sat \cT_\phi$.
%
Because of the De Morgan duality $\phi_1\vee \phi_2 \equiv \neg(\neg\phi_1\wedge \neg\phi_2)$, in fact we only have to provide constructions (of the second kind) for $\wedge$, $\neg$ and $\exists$, as well as for the weakening rule [ctx-weak] .

\begin{definition}[from atomic formulas to nested conditions]
\label{def:atomic formulas}
The following clauses define for each axiom of \Cref{def:formula in context} a corresponding proper nested condition, having the context as root interface. 
\begin{itemize}
\item $\cT_\True$ is given by $V=\setof v$ with $I_v=P_v=\tupof\emptyset$ (the empty graph), where $u_v=d_v=\id_{I_v}$.
\item $\cT_\False$ is given by $V=\setof{v,w}$ with $v\parentof w$, $I_v=P_v=I_w=P_w=\tupof\emptyset$, where $u_v=d_v=u_w=d_w=\id_{I_v}$.
\item $\cT_{\la(x_1,x_2)}$ is given by $V=\setof{v}$ with $I_v=\tupof{\setof{x_1,x_2}}$ and $P_v$ having as nodes $\{x'_1, x'_2\}$ and a single $\la$-labelled edge from $x'_1$ to $x'_2$. Furthermore $u_v=\id_{I_v}$ and $d_v\of I_v\to P_v$ is the morphism mapping $x_i$ to $x'_i$ for $i \in \setof{1,2}$. 
\item $\cT_{x_1\Eq x_2}$ is given by $V=\setof v$ with $I_v=\tupof{\setof{x_1,x_2}}$ and $P_v=\tupof{\setof{y}}$, where $u_v=\id_{I_v}$ and $d_v\of I_v\to P_v$ is the only available morphism.
\end{itemize}
\end{definition}

\begin{proposition}
Let $\cT$ be a {\proper} nested condition and $g\of I_\cT\to G$ a graph morphism.
\begin{enumerate}
\item If $\cT=\cT_\True$, then $g\sat \cT$ if and only if $g\sat\True$.
\item If $\cT=\cT_\False$, then $g\sat \cT$ if and only if $g\sat\False$.
\item If $\cT=\cT_{\la(x_1,x_2)}$, then $g\sat \cT$ if and only if $g\sat\la(x_1,x_2)$.
\item If $\cT=\cT_{x_1\Eq x_2}$, then $g\sat \cT$ if and only if $g\sat x_1\Eq x_2$.
\end{enumerate}
\end{proposition}

To encode negation, according to rule [ctx-$\neg$] given a {\proper} nested condition $\cT$ for the formula in context $\ctx{X}{\phi}$, we must provide a {\proper} nested condition for $\ctx{X}{\neg \phi}$, thus having the same interface $\tupof{X}$. We introduce a construction $\neg\cT$ that essentially ``pushes'' $\cT$ one level down by introducing a new root, suitably renaming the nodes of the old root.
More explicitly, $\cU=\neg\cT$ is given by $V_\cU=\tupof{V_\cT\uplus\setof{rt}}$\footnote{The notation means that we assume to choose a node identity $rt$ not appearing in $V_\cT$.} with ${\parentof_\cU}= {\parentof_\cT}\cup \setof{(rt,\rtof\cT)}$, $I_{rt} = I_\cT$, $u_{rt}=\id_{I_\cT}$, $P_{rt} = \tupof{Z}$ with $Z$ a set of fresh node identities with $i: N_{I_\cT} \to Z$ a bijection, $d_{rt} = i$,  and $u_u$ for all $rt_\cT \parentof u$ adjusted by post-composing them with $i$, whereas for all remaining $v\in V_\cT$, $d_v$ and $u_v$ remain as they were in $\cT$.

\begin{proposition}
Let $\cT$ be a {\proper} nested condition and $g\of I_\cT\to G$ a graph morphism; then $g\sat \neg\cT$ if and only if $g\nsat\cT$.
\end{proposition}

To encode conjunction, according to rule [ctx-$wedge$], given two {\proper} nested conditions $\cT_1$ and $\cT_2$ for the formulas in context $\ctx{X}{\phi_1}$ and $\ctx{X}{\phi_1}$, therefore with the same root interface $\tupof{X}$, we must provide a {\proper} nested condition having the same root interface for the conjunction $\phi_1 \wedge \phi_2$. This construction is based on the pushout of the root morphisms. 

Let $\cT_i$ for $i=1,2$ be {\proper} nested conditions with $V_1\cap V_2= \emptyset $ and $I_{\cT_1}=I_{\cT_2}=I$, and denote $P_i=P_{rt_{\cT_i}}$ and $d_i=d_{\rt_{\cT_i}}$ for $i=1,2$. The pushout of $d_1$ and $d_2$ consists of a pushout object $P=P_1\times_I P_2$ and morphisms $d'_i\of P_i\to P$ for $i=1,2$, and for every morphism $f\of A\to P_i$ there is an extended morphism $f;d'_i\of A\to P$; in particular, we denote $d=d_1;d'_1=d_2;d'_2\of I\to P$.  Now we define $\cU=\cT_1\times \cT_2$ with $V_\cU=((V_1\setminus \setof{rt_{\cT_1}})\cup (V_2\setminus \setof{rt_{\cT_2}})) \uplus \setof{rt}$, 
${\parentof}_{\cU}= ({\parentof}_{\cT_1}\cup {\parentof}_{\cT_2})\restr (V_\cU \times V_\cU) \cup \{(rt, v) \mid v \childof_1 rt_{\cT_1} \vee v \childof_2 rt_{\cT_2}\}$ and
%
\begin{itemize}
\item $I^\cU_{\rt}=I$, $P^\cU_{\rt}=P$ and $d^\cU_{\rt}=d$;
\item $u^\cU_w=u^{\cT_i}_w;d'_i$ for all $\rt \parentof_\cU w$.
\end{itemize}

\begin{proposition}
Let $\cT_1,\cT_2$ be {\proper} nested conditions with  $V_1\cap V_2=\emptyset$ and $I_{\cT_1}=I_{\cT_2}=I$. For all $g\of I\to G$, $g\sat \cT_1\times \cT_2$ if and only if $g\sat\cT_1$ and $g\sat \cT_2$.
\end{proposition}
%

A formula in context can be obtained by weakening, using rule [ctx-weak]. If the premises of the rule hold, namely $\ctx{X}{\phi}$ is a formula in context, $X\subseteq Y$ and $\bv{\phi} \cap Y = \emptyset$ we know that the variables in $Y \setminus X$ do not appear neither free nor bound in $\phi$. Therefore given a {\proper} nested condition for $\ctx{X}{\phi}$, we can obtain one for $\ctx{X}{\phi}$ with the following auxiliary construction. For any discrete graph $\tupof{Z}$ disjoint from $\cT$, $\cU=\cT\oplus \tupof{Z}$ is equal to $\cT$ except that $\tupof{Z}$ is ``added'' to the root; hence
\begin{itemize}
\item $I^\cU_{\rt}=I^\cT_{\rt}\cup \tupof{Z}$ 
\item $P^\cU_{\rt}=P^\cT_{\rt}\cup \tupof{Z'}$, where $Z'$ are fresh node indentifiers with a bijection $i : Z \to Z'$. Let  $\epsilon$ be the embedding $ \epsilon \of P^\cT_{\rt}\to P^\cU_{\rt}$. 
\item $d^\cU_{\rt}=d^\cT_{\rt}\cup i$;
\item $u^\cU_w=u^\cT_w;\epsilon$ for all $w\childof \rt$. 
\end{itemize}

\begin{proposition}
Let $\cT$ be a {\proper} nested condition and $\tupof{Z}$ a discrete graph disjoint from $\cT$. Then $\cT \oplus \tupof{Z}$ is a {\proper} nested condition and for all $g\of I_{\rt}\cup \tupof{Z}\to G$, $g\sat \cT\oplus \tupof{Z}$ if and only if $g\setminus \tupof{Z} \sat\cT$.
\end{proposition}



% The following proposition shows that we can now encode FOL conjunction.

% \begin{proposition}
% Let $\cT_1,\cT_2$ be nested conditions with $\rt=\rt_1=\rt_2$ and $V_1\cap V_2=\setof{\rt}$, and let $J=I_{\cT_1}\setminus I_{\cT_2}$ and $J_2=I_{\cT_2}\setminus I_{\cT_1}$. Assume that $J_1$ and $P_{\cT_2,\rt}$ as well as $J_2$ and $P_{\cT_1,\rt}$ are disjoint. For all $g\of (I_{\cT_1}\cup I_{\cT_2})\to G$, $g\sat (\cT_1\otimes J_2)\times (\cT_2\otimes J_1)$ if and only if $g\setminus J_2\sat\cT_1$ and $g\setminus J_1\sat \cT_2$.
% \end{proposition}

%=======================


