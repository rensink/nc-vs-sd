\section{Definitions}
\label{sec:definitions}

\begin{itemize}
\item An \emph{interface} $I$ is a discrete graph.

\item An \emph{open graph} is an arrow $g\of I\to G$ where $I$ is an interface; we say that $g$ has \emph{interface} $I_g=I$ and \emph{pattern} $P_g=G$. Open graphs are used for different purposes: within a condition to associate (both upper and lower) interfaces with the patterns, and also as assignments --- that is, the objects on which satisfaction is defined.

\item An \emph{interface morphism} $f$ from $I$ to $J$ is a graph morphism $k:I\to J$.

\item An \emph{open graph morphism} $a\of g\to h$ a pair of graph morphisms $a^I:I_g\to I_h, a^P:P_g\to P_h$ such that $a^I;h=g;a^P$.
\end{itemize}
%
\begin{definition}[nested condition]\label{def:nested condition}
A \emph{nested condition} is a tuple $(V,E,\setof{d_v}_{v\in V},\setof{u_w}_{(v,w)\in C})$ such that

\begin{itemize}[topsep=\smallskipamount]
\item $V$ is a set of vertices;

\item $C\subseteq V\times V$ is a set of edges representing the parent-child relation, such that one $v\in P$ (the \emph{root}) has no outgoing edge, and all other elements of $V$ have exactly one outgoing edge (to their \emph{parent});

\item for all $v\in V$, $d_v$ is an open graph. We denote $I_v=I_{d_v}$ and $P_v=P_{d_v}$.

\item for all $c=(v,w)\in C$, $u_w\of I_w\to P_v$ is an open graph.
\end{itemize}
\end{definition}
%
The root of a nested condition $\cT$ is denoted $\rt\cT$; its \emph{root interface} is $I_\cT=I_{\rt\cT}$. For an arbitrary vertex $v\in V_\cT$, $\hat v$ will denote the subtree rooted at $v$. We often identify the triple $(v,a,w)$ with the open graph morphism $a$, when $v$ and $w$ are clear from the context.

\begin{definition}[satisfaction]\label{def:satisfaction}
Let $\cT$ be a condition tree and $g\of I^v\to G$ an assignment. $g$ \emph{satisfies $v\in V_\cT$}, denoted $g\sat v$, if there is a graph morphism $h\of P^v\to g$ such that
\begin{inumerate}
\item $g=d_v;h$, and 
\item $u_w;a_P\nsat w$ for all children $w$ of $v$.
\end{inumerate}
$h$ is then called a \emph{witness} of $g\sat v$.
$g$ satisfies $\cT$, denoted $g\sat\cT$, if $g\sat\rt\cT$.
\end{definition}

\paragraph{Morphisms.}

\begin{definition}
Given two condition trees $\cT,\cU$ with $I_\cT=I_\cU$, a morphism $\cM$ from $\cT$ to $\cU$ is a set of triples $(v,a,w)$ where $(v,w)\in (V_\cT\times V_\cU)\cup (V_\cU\times V_\cT)$ and $a\of d_v\to d_w$ is an open graph morphism, such that $\rt\cM=(\rt\cT,(\id,r),\rt\cU)\in \cM$ for some $r$.
\end{definition}
%
We often write $a\of v\to_\cM w$ rather than $(v,a,w)\in \cM$.

Given a condition tree $\cT$, the identity morphism $\cI_\cT$ is defined as
\[ \cI_\cT = \setof{(v,\id_{d_v},v)\mid v\in V_\cT} \enspace. \]
Given two condition tree moprhisms $\cM\of \cT\to \cU,\cN\of \cU\to \cV$, their composition is defined as
\[ \cM;\cN =
\begin{array}[t]{@{}l}
	\setof{(v,a;b,x) \mid \exists w\st a\of v\to_\cM w, b\of w\to_\cN x} \\
	\cup \setof{(v,a;b,x) \mid \exists w\st a\of v\to_\cN w, b\of w\to_\cM x} \enspace.
\end{array}
\]

\begin{proposition}
Condition trees and their morphisms form a category.
\end{proposition}

We are interested in nested condition morphisms that either preserve or reflect satisfaction, in the following sense.

\begin{definition}
Let $\cM\of \cT\to\cU$ be a morphism and let $a\of v\to_\cM w$.
\begin{itemize}[topsep=\smallskipamount]
\item $a$ \emph{preserves satisfaction} if $a^I;g\sat v$ with witness $h$, where either $w=\rt\cU$ or $g=u_w;g'$ for some $g'$, implies $g\sat w$ with witness $h'$ such that $h=a^P;h'$. $\cM$ preserves satisfaction if $\rt\cM$ preserves satisfaction.
\item $a$ \emph{reflects satisfaction} if $g\sat w$ with witness $h$ implies $a^I;g\sat v$ with witness $a^P;h$. $\cM$ preserves satisfaction if $\rt\cM$ preserves satisfaction.
\end{itemize}
\end{definition}

\paragraph{Saturation.}

To get the right interaction between morphisms and satisfaction, the morphisms have to be \emph{saturated} in a certain sense: a triple $(v,a,w)$ preserves or reflects satisfaction if the morphism contains \emph{supporting} mappings for all children of $w$, respectively all children of $v$. To set up a rich enough framework, we need diverse notions of support.

\begin{definition}[source support]\label{def:support}
Let $\cM\of \cT\to\cU$ be an NC-morphism. For a given $a\of v\to_\cM w$ and child $x$ of $v$:
\begin{itemize}[topsep=\smallskipamount]
\item \emph{(Direct source support.)} $\issdir b{x,a}$ if $b\of y\to_\cM x$ for some child $y$ of $w$ such that $b^I;u_x;a^P=u_y$.
\item \emph{(Child-based source support.)} $\isschd b{x,a}$ if $b\of y\to_\cM w$ for some child $y$ of $x$ such that $k;u_x;a^P=d_y;b^P$ for some $k\of I_y\to I_x$.
\item \emph{(Sibling-based source support.)} $\isssib b{x,a}$ if $b\of y\to_\cM x$ for some sibling $y$ of $v$ such that $k;d_x=b^I;u_w;a^P$ for some $k\of I_y\to I_w$.
\end{itemize}
\end{definition}
%
For a visualisation see \cref{fig:source-saturation}.
%
\begin{figure}
	\subcaptionbox{
  Standard
}[.25\textwidth]{
\begin{tikzpicture}[on grid,baseline=(Iv)]
\node (Iv) {};
\node (Pv) [below=of Iv] {};
\node (Ix) [below=of Pv] {};
\node (Px) [below=of Ix] {};

\path
  (Iv) edge[->] node[opengraph] {} node[left] {$d_v$} (Pv)
  (Ix) edge[->] node[opengraph] {} node[left] {$d_x$} (Px)
  (Ix) edge[->] node[left] {$u_x$} (Pv);
  
\node (Iw) [right=1.5 of Iv] {};
\node (Pw) [below=of Iw] {};

\path
  (Iw) edge[->] node[opengraph] {} node[right] {$d_w$} (Pw);

\path[morphism]
  (Iv) edge[->] node[above] {$a^I$} (Iw)
  (Pv) edge[->] node[above] {$a^P$} (Pw);

\node (Iy) [below=of Pw] {};
\node (Py) [below=of Iy] {};

\path[allcolor=gray]
  (Iy) edge[->] node[opengraph] {} node[right] {$d_y$} (Py)
  (Iy) edge[->] node[right] {$u_y$} (Pw);

\path[morphism=blue!40!white]
  (Iy) edge[->] node[above] {$b^I$} (Ix)
  (Py) edge[->] node[above] {$b^P$} (Px);

% Defs to make sure bottom is aligned
\node (I) [below=1 of Px] {};
\node (P) [below=1 of I] {};
\path[draw=none]
  (I) to node[opengraph,fill=none] {} (P);
\end{tikzpicture}}%
%
\subcaptionbox{
  Child rule
}[.4\textwidth]{
\begin{tikzpicture}[on grid,baseline=(Iv)]
\node (Iv) {};
\node (Pv) [below=of Iv] {};
\node (Ix) [below=of Pv] {};
\node (Px) [below=of Ix] {};

\path
  (Iv) edge[->] node[opengraph] {} node[left] {$d_v$} (Pv)
  (Ix) edge[->] node[opengraph] {} node[left] {$d_x$} (Px)
  (Ix) edge[->] node[left] {$u_x$} (Pv);
  
\node (Iw) [right=2.5 of Iv] {};
\node (Pw) [below=of Iw] {};

\path
  (Iw) edge[->] node[opengraph] {} node[right] {$d_w$} (Pw);

\path[morphism]
  (Iv) edge[->] node[above] {$a^I$} (Iw)
  (Pv) edge[->] node[above] {$a^P$} (Pw);

\node (Iy) [below=of Px] {};
\node (Py) [below=of Iy] {};

\path[allcolor=black!50!white]
  (Iy) edge[->] node[opengraph] {} node[left] {$d_y$} (Py)
  (Iy) edge[->] node[left,pos=.6,inner sep=1] {$u_y$} (Px);

% The pullback
\node (Ixy) [left=1.5 of Px] {};
\path[allcolor=black!50!white]
  (Ixy) edge[->] node[above left] {$u$} (Ix.west)
  (Ixy) edge[->>] node[below left] {$d$} (Iy.west)
  (Ixy) edge[-{Straight Barb[black!50!white,length=5pt,width=10pt]},white] +(5mm,0mm);

\path[morphism=blue!40!white]
  (Iy.east) edge[->] node[left] {$b^I$} (Iw)
  (Py.east) edge[->] node[right] {$b^P$} (Pw);
\end{tikzpicture}
}%
%
\subcaptionbox{
  Sibling rule
}[.35\textwidth]{
\begin{tikzpicture}[on grid,baseline=(Iv)]
\node (Iv) {};
\node (Pv) [below=of Iv] {};
\node (Iw) [below=of Pv] {};
\node (Pw) [below=of Iw] {};

\path
  (Iv) edge[->] node[opengraph] {} node[left] {$d_v$} (Pv)
  (Iw) edge[->] node[opengraph] {} node[left] {$d_w$} (Pw)
  (Iw) edge[->] node[left] {$u_w$} (Pv);
  
\node (Ix) [right=1.5 of Iv] {};
\node (Px) [below=of Ix] {};

\path
  (Ix) edge[->] node[opengraph] {} node[right] {$d_x$} (Px);

\path[morphism]
  (Iv) edge[->] node[above] {$a^I$} (Ix)
  (Pv) edge[->] node[above] {$a^P$} (Px);

\node (Iy) [right=1.5 of Ix] {};
\node (Py) [below=of Iy] {};
\node (Pz) [above right=.75 and .75 of Ix] {};

\path[allcolor=gray]
  (Iy) edge[->] node[opengraph] {} node[right] {$d_y$} (Py)
  (Ix) edge[->] node[left] {$u_x$} (Pz)
  (Iy) edge[->] node[right] {$u_y$} (Pz)
  (Iy) edge[->] node[above] {$k$} (Ix);

\path[morphism=blue!40!white]
  (Iy.west) edge[->,bend left] node[below] {$b^I$} (Iw)
  (Py.south) edge[->,bend left] node[below] {$b^P$} (Pw);

% Defs to make sure bottom is aligned
\node (I) [below=1 of Pw] {};
\node (P) [below=1 of I] {};
\path[draw=none]
  (I) to node[opengraph,fill=none] {} (P);
\end{tikzpicture}
}%
	\caption{The notions of support of \cref{def:support}}
	\label{fig:source-saturation}
\end{figure}
%
Based on these notions of support, we define corresponding notions of \emph{saturation}:

\begin{definition}[syntactic reflection]\label{def:syntactic reflection}
Let $\cM\of \cT\to\cU$ be an NC-morphism, and let $a\of v\to_\cM w$.
\begin{itemize}[topsep=\smallskipamount]
\item \emph{(Direct reflection.)} $a:\rdir$ if for every child $x$ of $v$, there is some $b:\rdir$ such that $\issdir b{x,a}$.
\item \emph{(Child-based reflection.)} $a:\rchd$ if for every child $x$ of $v$, there is some $b:\rchd$ such that $\issdir b{x,a}$ or $\isschd b{x,a}$.
\item \emph{(Sibling-based reflection.)} $a:\rsib$ if for every child $x$ of $v$, there is some $b:\rsib$ such that $\issdir b{x,a}$ or $\isssib b{x,a}$.
\end{itemize}
\end{definition}

Vacuously, each of the conditions of reflection is fulfilled if $v$ does not have children, hence in that case all $a\of v\to_\cM x$ satisfy $\rdir$, $\rchd$ and $\rsib$; this will form the base case of our induction proofs.

\begin{lemma}[source saturation reflects satisfaction]\label{lem:source-saturation}
Let $\cM\of\cT\to\cU$ be a condition tree morphism and let $a\of v\to_\cM w$.
\begin{enumerate}[topsep=\smallskipamount]
\item $a:\rdir$ implies that $a$ reflects satisfaction. 
\item $a:\rchd$ implies that $a$ reflects satisfaction. 
\item $a:\rsib$ implies that $a$ reflects satisfaction. 
\end{enumerate}
\end{lemma}

\paragraph{Target saturation.}

\begin{definition}[target support]\label{def:target support}
Let $\cM\of \cT\to\cU$ be an NC-morphism, let $a\of v\to_\cM w$ and let $y$ be a child of $w$.
\begin{itemize}[topsep=\smallskipamount]
\item \emph{(Direct target support.)} $b:\tdir(y,a)$ if 
\begin{inumerate}
\item either $a=\id$ or $w$ is not the root of $\cU$, and there is an arrow $a'\of P_v\to P$, with $P$ the pushout object of $(u_w,d_w)$, and
\item $b\of y\to_\cM x$ with $x$ a child of $v$.
\end{inumerate}
\end{itemize}
\end{definition}

\begin{definition}[syntactic preservation]
Let $\cM\of \cT\to\cU$ be an NC-morphism, and let $a\of v\to_\cM w$.
\begin{itemize}[topsep=\smallskipamount]
\item \emph{(Direct preservation.)} $a:\pdir$ if for every child $y$ of $w$, there is some $b:\tdir(y,a)$ such that $b:\pdir$ .
\item \emph{(Child-based preservation.)} $a:\pchd$ if one of the following holds:
\begin{itemize}
\item for every child $y$ of $w$, there is some $b:\tdir(y,a)$ such that $b:\pchd$;
\item there is a child $x$ of $v$ with an open graph morphism $c\of d_x\to d_v$, such that for every child $y$ of $x$ there is some $b\of y \to_\cM w$ with $b:\pchd$.
\end{itemize}
\end{itemize}
\end{definition}

