\section{Definitions}
\label{sec:definitions}

\begin{itemize}
\item An \emph{interface} $I$ is a discrete graph.

\item An \emph{open graph} is an arrow $g\of I\to G$ where $I$ is an interface; we say that $g$ has \emph{interface} $I_g=I$ and \emph{pattern} $P_g=G$. Open graphs are used for different purposes: within a condition to associate (both upper and lower) interfaces with the patterns, and also as assignments --- that is, the objects on which satisfaction is defined.

\item An \emph{interface morphism} $f$ from $I$ to $J$ is a graph morphism $k:I\to J$.

\item An \emph{open graph morphism} $a\of g\to h$ a pair of graph morphisms $a^I:I_g\to I_h, a^P:P_g\to P_h$ such that $a^I;h=g;a^P$.
\end{itemize}
%
\begin{definition}[condition tree]\label{def:condition tree}
A \emph{condition tree} is a tuple $\cT=\tupof{V, {\parentof}, \setof{d_v}_{v\in V}, \setof{u_w}_{(v,w)\in C}}$ such that

\begin{itemize}[topsep=\smallskipamount]
\item $V$ is a set of vertices;

\item ${\parentof}\subseteq V\times V$ is a parent-child relation, such that one $v\in P$ (the \emph{root}) has no predecessor, and all other elements of $V$ have exactly one predecessor (their \emph{parent});

\item for all $v\in V$, $d_v$ is an open graph. We denote $I_v=I_{d_v}$ and $P_v=P_{d_v}$.

\item for all $c=(v,w)\in C$, $u_w$ is a graph morphism $I_w\to P_c$.
\end{itemize}
\end{definition}
%
We use $\ancestof$ and $\ancesteq$ to denote the transitive, respectively transitive and reflexive, closure of $\parentof$. The root of a condition tree $\cT$ is denoted $\rt\cT$; we denote $I_\cT=I_{\rt\cT}$, $P_\cT=P_{\rt\cT}$ and $d_\cT=d_{\rt\cT}$. For an arbitrary vertex $v\in V_\cT$, $\hat v$ will denote the subtree rooted at $v$, and if $v\neq\rt\cT$ then $\parent v$ denotes $v$'s parent.

\begin{definition}[satisfaction]\label{def:satisfaction}
Let $\cT$ be a condition tree and $g\of I_\cT\to G$ an assignment. $g$ \emph{satisfies $\cT$} denoted $g\sat \cT$, if there is a graph morphism $h\of P_\cT\to G$ such that
\begin{inumerate}
\item $g=d_\cT;h$, and 
\item $u_v;h\nsat \hat v$ for all children $v$ of $\rt\cT$.
\end{inumerate}
$h$ is then called a \emph{witness} of $g\sat \cT$.
\end{definition}
%
If $g$ satisfies $\cT$, we also call $g$ a $\cT$-model.

\todo[inline]{At this point introduce the connection from condition trees to FOL}

\begin{definition}[morphism]
Given two condition trees $\cT,\cU$ with $I_\cT=I_\cU=I$, a morphism $\cM$ from $\cT$ to $\cU$ is a set of triples $(v,a,w)$ where $(v,w)\in (V_\cT\times V_\cU)\cup (V_\cU\times V_\cT)$ and $a\of d_v\to d_w$ is an open graph morphism. $\cM$ has a distinguished element $\rt\cM$, called the \emph{root}, of the shape $\rt\cM=(\rt\cT,(\id_I,r),\rt\cU)\in \cM$ for some $r$.
\end{definition}
%
We typically denote a triple $(v,a,w)$ of the kind above as $\trip vaw$. Moreover, we also sometimes use the open graph morphism $a$ to refer to the triple $\trip vaw$, when that causes no confusion and the identity of $v$ and $w$ is clear from the context.

We use complete induction to reason about condition trees and condition tree morphisms. Induction will be based on the \emph{depth} of node-induced subtrees. For a given $\cT$,\todo{AR: I inserted this, but it's not clear to me that we need it.}
%
\[ \depth_\cT: v\mapsto \max \gensetof{\depth_\cT(w)}{v\childof w} \enspace. \]


\medskip\noindent
Given a condition tree $\cT$, the identity morphism $\cI_\cT\of \cT\to\cT$ is defined as
\[ \cI_\cT = \setof{\trip v{\id_{d_v}}v \mid v\in V_\cT} \enspace. \]
Given two condition tree morphisms $\cM\of \cT\to \cU,\cN\of \cU\to \cV$, their composition $\cM;\cN\of \cT\to \cV$ is defined as
\[ \cM;\cN =
\begin{array}[t]{@{}l}
	\setof{\trip v{a;b}x \mid \exists w\st \trip v a w\in\cM, \trip wbx\in \cN} \\
	\cup \setof{\trip v{a;b}x \mid \exists w\st \trip vaw\in\cN, \trip wbx\in\cM} \enspace.
\end{array}
\]

\begin{proposition}
Condition trees and their morphisms form a category, denoted {\CT}.
\end{proposition}
%
We are interested in condition tree morphisms that either preserve or reflect models. In order to reason about this (inductively), we have to define the notions of model preservation and reflection not just on the level of morphisms, but for arbitrary triples $\trip vaw$ in those morphisms. For such a triple, in general $I_v$ and $I_w$ do not coincide. This means that the model sets of $\hat v$ and $\hat w$, being graph morphisms sourced at $I_v$ and $I_w$, are typically disjoint. Instead the model sets can only be expected to be related modulo $a^I$, in the sense that we relate candidate $\hat v$-models $g_v:I_v\to G$ to candidate $\hat w$-models $g_w:I_w\to G$ for which $g_v=a^I;g_w$.\footnote{In a previous paper \cite{RensCorr}, we introduced the terms \emph{reflection while adding $a^I$} and \emph{preservation while subtracting $a^I$}.}

\subsection{Model reflection}

With the above in mind, we first define reflection of models, this being the more straightforward of the two directions.

\begin{definition}[model reflection]
Let $\cM\of \cT\to\cU$ be a morphism and let $\trip vaw\in\cM$.
\begin{itemize}
\item $a$ \emph{reflects} a $\hat w$-model $g$ with witness $h$ if $a^I;g\sat \hat v$ with witness $a^P;h$.

\item $a$ \emph{reflects models} if $a$ reflects all $\hat w$-model/witness pairs.

\item $\cM$ reflects models if $\rt\cM$ reflects models.
\end{itemize}
\end{definition}
%
A given morphism can be shown to reflect models if it provides enough internal \emph{evidence} for reflection: namely, if for a candidate model-reflecting triple $\trip vaw\in \cM$ (i.e., for which we want to show that every $\hat w$-model/witness pair $g,h$ gives rise to a $\hat v$-model/witness pair $a^I;g,a^P;h$) and every source child $x$ of $v$, there is further structure in $\hat v$ and $\cM$ that ensure that $u_x;a^P;h\nsat \hat x$. To obtain a rich enough framework, we introduce diverse notions of evidence.

\begin{definition}[reflection evidence]\label{def:support}
Let $\cM\of \cT\to\cU$ be an CT-morphism. For a given $\trip vaw\in\cM$ and child $x$ of $v$, we define the following kinds of \emph{reflection evidence}.
\begin{itemize}[topsep=\smallskipamount]
\item \emph{(Direct reflection evidence.)} $\trip ybx\in\cM$ provides \emph{direct reflection evidence}, denoted $\issdir b{x,a}$, if $w\parentof y$ and $b^I;u_x;a^P=u_y$.
\item \emph{(Child-based reflection evidence.)} $\trip ybw\in\cM$ provides \emph{child-based reflection evidence}, denoted $\isschd b{x,a}$, if $x\parentof y$ and $k;u_x;a^P=d_y;b^P$ for some $k\of I_y\to I_x$.
\item \emph{(Sibling-based reflection evidence.)} $\trip ybx\in\cM$  provides \emph{sibling-based reflection evidence}, denoted $\isssib b{x,a}$, if $\parent x\parentof y$ and $k;d_w=b^I;u_x;a^P$ for some $k\of I_y\to I_w$.
\end{itemize}
\end{definition}
%
Alternatively: for all $\trip vaw\in\cM$ and child $x$ of $v$, we can define the sets $\sdir{a,x}$, $\schd{a,x}$ and $\ssib{a,x}$ as below, and the set $\re{a,x}$ of (generalised) reflection evidence as their union.
%
\begin{align*}
\sdir{a,x} ={} & \gensetof{\trip ybx\in \cM}{w\parentof y, b^I;u_x;a^P=u_y} \\
\schd{a,x} ={} & \gensetof{\trip ybw\in \cM}{x\parentof y, \exists k:I_y\to I_x\st k;u_x;a^P=d_y;b^P} \\
\ssib{a,x} ={} & \gensetof{\trip ybx\in \cM}{\parent v\parentof y, \exists k:I_y\to I_x\st k;d_w=b^I;u_x;a^P} \\
\re{a,x} ={} & \sdir{a,x}\cup \schd{a,x} \cup \ssib{a,x} \enspace.
\end{align*}
%
For a visualisation see \cref{fig:source-support}.
%
\begin{figure}
	\subcaptionbox{
  Standard
}[.25\textwidth]{
\begin{tikzpicture}[on grid,baseline=(Iv)]
\node (Iv) {};
\node (Pv) [below=of Iv] {};
\node (Ix) [below=of Pv] {};
\node (Px) [below=of Ix] {};

\path
  (Iv) edge[->] node[opengraph] {} node[left] {$d_v$} (Pv)
  (Ix) edge[->] node[opengraph] {} node[left] {$d_x$} (Px)
  (Ix) edge[->] node[left] {$u_x$} (Pv);
  
\node (Iw) [right=1.5 of Iv] {};
\node (Pw) [below=of Iw] {};

\path
  (Iw) edge[->] node[opengraph] {} node[right] {$d_w$} (Pw);

\path[morphism]
  (Iv) edge[->] node[above] {$a^I$} (Iw)
  (Pv) edge[->] node[above] {$a^P$} (Pw);

\node (Iy) [below=of Pw] {};
\node (Py) [below=of Iy] {};

\path[allcolor=gray]
  (Iy) edge[->] node[opengraph] {} node[right] {$d_y$} (Py)
  (Iy) edge[->] node[right] {$u_y$} (Pw);

\path[morphism=blue!40!white]
  (Iy) edge[->] node[above] {$b^I$} (Ix)
  (Py) edge[->] node[above] {$b^P$} (Px);

% Defs to make sure bottom is aligned
\node (I) [below=1 of Px] {};
\node (P) [below=1 of I] {};
\path[draw=none]
  (I) to node[opengraph,fill=none] {} (P);
\end{tikzpicture}}%
%
\subcaptionbox{
  Child rule
}[.4\textwidth]{
\begin{tikzpicture}[on grid,baseline=(Iv)]
\node (Iv) {};
\node (Pv) [below=of Iv] {};
\node (Ix) [below=of Pv] {};
\node (Px) [below=of Ix] {};

\path
  (Iv) edge[->] node[opengraph] {} node[left] {$d_v$} (Pv)
  (Ix) edge[->] node[opengraph] {} node[left] {$d_x$} (Px)
  (Ix) edge[->] node[left] {$u_x$} (Pv);
  
\node (Iw) [right=2.5 of Iv] {};
\node (Pw) [below=of Iw] {};

\path
  (Iw) edge[->] node[opengraph] {} node[right] {$d_w$} (Pw);

\path[morphism]
  (Iv) edge[->] node[above] {$a^I$} (Iw)
  (Pv) edge[->] node[above] {$a^P$} (Pw);

\node (Iy) [below=of Px] {};
\node (Py) [below=of Iy] {};

\path[allcolor=black!50!white]
  (Iy) edge[->] node[opengraph] {} node[left] {$d_y$} (Py)
  (Iy) edge[->] node[left,pos=.6,inner sep=1] {$u_y$} (Px);

% The pullback
\node (Ixy) [left=1.5 of Px] {};
\path[allcolor=black!50!white]
  (Ixy) edge[->] node[above left] {$u$} (Ix.west)
  (Ixy) edge[->>] node[below left] {$d$} (Iy.west)
  (Ixy) edge[-{Straight Barb[black!50!white,length=5pt,width=10pt]},white] +(5mm,0mm);

\path[morphism=blue!40!white]
  (Iy.east) edge[->] node[left] {$b^I$} (Iw)
  (Py.east) edge[->] node[right] {$b^P$} (Pw);
\end{tikzpicture}
}%
%
\subcaptionbox{
  Sibling rule
}[.35\textwidth]{
\begin{tikzpicture}[on grid,baseline=(Iv)]
\node (Iv) {};
\node (Pv) [below=of Iv] {};
\node (Iw) [below=of Pv] {};
\node (Pw) [below=of Iw] {};

\path
  (Iv) edge[->] node[opengraph] {} node[left] {$d_v$} (Pv)
  (Iw) edge[->] node[opengraph] {} node[left] {$d_w$} (Pw)
  (Iw) edge[->] node[left] {$u_w$} (Pv);
  
\node (Ix) [right=1.5 of Iv] {};
\node (Px) [below=of Ix] {};

\path
  (Ix) edge[->] node[opengraph] {} node[right] {$d_x$} (Px);

\path[morphism]
  (Iv) edge[->] node[above] {$a^I$} (Ix)
  (Pv) edge[->] node[above] {$a^P$} (Px);

\node (Iy) [right=1.5 of Ix] {};
\node (Py) [below=of Iy] {};
\node (Pz) [above right=.75 and .75 of Ix] {};

\path[allcolor=gray]
  (Iy) edge[->] node[opengraph] {} node[right] {$d_y$} (Py)
  (Ix) edge[->] node[left] {$u_x$} (Pz)
  (Iy) edge[->] node[right] {$u_y$} (Pz)
  (Iy) edge[->] node[above] {$k$} (Ix);

\path[morphism=blue!40!white]
  (Iy.west) edge[->,bend left] node[below] {$b^I$} (Iw)
  (Py.south) edge[->,bend left] node[below] {$b^P$} (Pw);

% Defs to make sure bottom is aligned
\node (I) [below=1 of Pw] {};
\node (P) [below=1 of I] {};
\path[draw=none]
  (I) to node[opengraph,fill=none] {} (P);
\end{tikzpicture}
}%
	\caption{The notions of support of \cref{def:support}}
	\label{fig:source-support}
\end{figure}
%
Based on these notions of evidence, we define corresponding notions of \emph{syntactic reflection}:

\begin{definition}[syntactic reflection]\label{def:syntactic reflection}
Let $\cM\of \cT\to\cU$ be a condition tree morphism, and let $\trip vaw\in\cM$.
\begin{itemize}[topsep=\smallskipamount]
\item \emph{(Direct reflection.)} $a\in\rdir$ if for every child $x$ of $v$, there is some $b\in\rdir$ such that $\issdir b{x,a}$.
\item \emph{(Child-based reflection.)} $a\in\rchd$ if for every child $x$ of $v$, there is some $b\in\rchd$ such that $\issdir b{x,a}$ or $\isschd b{x,a}$.
\item \emph{(Sibling-based reflection.)} $a\in\rsib$ if for every child $x$ of $v$, there is some $b\in \rsib$ such that $\issdir b{x,a}$ or $\isssib b{x,a}$.
\item \emph{(General reflection.)} $s\in \rgen$ if for every child $x$ of $v$, there is some $b\in \rgen$ such that $\issdir b{x,a}$, $\isschd b{x,a}$ or $\isssib b{x,a}$.
\end{itemize}
\end{definition}
%
Alternatively: $\rdir$, $\rchd$, $\rsib$ and $\rgen$ are the smallest sets satisfying the following recursive equations:
\begin{align*}
\rdir ={} & \gensetof{\trip vaw\in \cM}{\forall x\childof v\st \rdir \cap \sdir{a,x}\neq\emptyset} \\
\rchd ={} &\gensetof{\trip vaw\in \cM}{\forall x\childof v\st \rchd \cap (\sdir{a,x}\cup \schd{a,x}) \neq\emptyset} \\
\rsib ={} &\gensetof{\trip vaw\in \cM}{\forall x\childof v\st \rsib \cap (\sdir{a,x}\cup \ssib{a,x}) \neq\emptyset} \\
\rgen ={} &\gensetof{\trip vaw\in \cM}{\forall x\childof v\st \rgen \cap \re{a,x} \neq\emptyset} \enspace.
\end{align*}
%
Clearly, $\rdir\subseteq \rchd\cap \rsib$ and $\rdir\cup \rchd \cup \rsib \subseteq \rgen$; also, if $v$ does not have children then $\trip vaw\in \rdir$ for all $\trip vaw\in\cM$. The latter observation will form the base case of our induction proofs.

\begin{lemma}[syntactic reflection implies model reflection]\label{lem:reflection}
If $\cM\of \cT\to \cU$ is a condition tree morphism, then $a$ reflects models for all $a\in\rgen_\cM$.
\end{lemma}
%
\begin{proof}
We consider the $\trip vaw\in \rgen_\cM$. Let $g\sat \hat w$; we set out to prove that $a^I;g\sat \hat v$. By way of induction hypothesis, we assume that this property holds for $\trip xyb\in \rgen_\cM$ when the . The base case is obtained when both $v$ and $w$ are leaves, and the induction is well-founded because both $\cT$ and $\cU$ have finite depth.

$g\sat \hat w$ implies that there is a witness $h$ such that $g=d_w;h$ and $u_z;h\nsat \hat z$ for all $z\childof w$. It follows that $a^I;g=a^I;d_w;h=d_v;a^P;h$, hence $a^P;h$ is a prospective witness for $a^I;g\sat \hat v$, and we are done if we can show that $u_x;a^P;h\nsat \hat x$ for all $x\childof v$. Now we use the fact that (by definition of $\rgen_\cM$) there is some $b\in \rgen \cap \re{a,x}$. The proof proceeds by contradiction, using a case distinction based on the kind of reflection evidence provided by $b$. Note that the induction hypothesis applies to $b$.
%
\begin{itemize}
\item $b\in \sdir[\cM]{a,x}$. It follows that $\trip ybx\in\cM$ such that $w\parentof y$ and $b^I;u_x;a^P=u_y$. Assume $u_x;a^P;h\sat \hat x$; then by the induction hypothesis, $b^I;u_x;a^P;h\sat \hat y$, implying $u_y;h\sat \hat y$, which contradicts $g\sat \hat w$.

\item $b\in \schd[\cM]{a,x}$. It follows that $\trip ybw\in \cM$ such that $x\parentof y$ and $k;u_x;a^P=d_y;b^P$ for some $k\of I_y\to I_x$. By the induction hypothesis, $b^I;g\sat \hat y$. Since $b^I;g=b^I;d_w;h=d_y;b^P;h= k;u_x;a^P;h$,  Assume $u_x;a^P;h\sat \hat x$

\item $b\in \ssib[\cM]{a,x}$. It follows that $\trip ybx\in \cM$ such that $\parent x\parentof y$ and $k;d_w=b^I;u_x;a^P$ for some $k\of I_y\to I_x$.
\end{itemize}
\end{proof}

\begin{corollary}
If $\cM$ is a condition tree morphism, then $\rt\cM\in \rgen_\cM$ implies that $\cM$ reflects models.
\end{corollary}

\subsection{Model preservation}

In order to reason concisely about model preservation, we call a graph morphism $f:I\to P$ \emph{$e$-prefixed}, for some graph morphism $e:I\to P'$, if $f=e;f'$ for some (not necessarily unique) graph morphism $f':P'\to P$.

To define model preservation, we have to take into account that (as observed earlier) a triple $\trip vaw$ cannot be expected to carry over any $\hat v$-model $g$ to some $\hat w$-model: rather, $g$ should at least be $a^I$-prefixed for this to make much sense. In fact, we define the notion of model preservation for $u$-prefixed models of $\hat v$, where $u$ is a parameter that can itself be instantiated by any $a^I$-prefixed morphism. (Such a $u$-prefixed model is thus certainly an $a^I$-prefixed model.) This additional complication is required to make our induction proofs work.
%
\begin{definition}[model preservation]
Let $\cM\of \cT\to\cU$ be a morphism and let $\trip vaw\in\cM$.
\begin{itemize}[topsep=\smallskipamount]
\item $a$ \emph{preserves} an $a^I$-prefixed $\hat v$-model $g=a^I;g'$ with witness $h$ if $h=a^P;h'$ for some $h'$ such that $g'\sat \hat w$ with witness $h'$.
		
\item Let $u:I_v\to P$ be an $a^I$-prefixed graph morphism. $a$ \emph{preserves $u$-prefixed models} if $a$ preserves all $u$-prefixed $\hat v$-model/witness pairs.
		
\item $\cM$ preserves models if $\rt\cM$ preserves $\id$-prefixed models.
\end{itemize}
\end{definition}
%
Like for model reflection, we prove model preservation for morphisms with a certain ``syntactic" structure.

\begin{definition}
Let $\cM\of \cT\to\cU$ be a condition tree morphism, and let $u\of I\times P$ be a graph morphism. A triple $\trip vaw\in\cM$ is a \emph{$u$-fusion} if $a^I$ is epi, $u$ is $a^I$-prefixed and the pushout of $u$ over $d_v$ is $a^P$-prefixed. The set of $u$-fusions in $\cM$ is denoted $\fuse(u)$.
\end{definition}

\begin{definition}[target support]\label{def:target support}
Let $\cM\of \cT\to\cU$ be a condition tree morphism, let $\trip vaw\in\cM$ and let $y$ be a child of $w$.
\begin{itemize}[topsep=\smallskipamount]
\item \emph{(Direct target support.)} $b:\tdir{y,a}$ if 
\begin{inumerate}
\item $v=\rt\cT$ and $a=\id$, or $v\neq\rt\cT$ and there is an arrow $a'\of P_w\to P$ with $P$ the pushout object of $(u_v,d_v)$; and
\item $\trip ybx\in \cM$ for some child $x$ of $v$, such that $b^I;u_x;a^P=u_y$.
\end{inumerate}
\end{itemize}
\end{definition}
%
For a visualisation see \cref{fig:target-support}.
%
\begin{figure}
	\subcaptionbox*{
	Direct target support \\ for root nodes \\ ($\istdir b{x,a}$)
}[.5\textwidth][c]{
\begin{tikzpicture}[on grid,baseline=(Iv)]
	\node (Iv) {};
	\node (Pv) [below=of Iv] {};
	\node (Ix) [below left=of Pv] {};
	\node (Px) [below=of Ix] {};
	
	\path
	(Iv) edge[->] node[opengraph] {} node[left] {$d_v$} node[right] {$d_w$} (Pv);
	
	\path[allcolor=gray]
	(Ix) edge[->] node[opengraph] {} node[left] {$d_x$} (Px)
	(Ix) edge[->] node[left] {$u_x$} (Pv);
	
	\node (Iy) [below right=of Pv] {};
	\node (Py) [below=of Iy] {};
	
	\path
	(Iy) edge[->] node[opengraph] {} node[right] {$d_y$} (Py)
	(Iy) edge[->] node[right] {$u_y$} (Pv);
	
	\path[morphism=blue!40!white]
	(Iy) edge[->] node[above] {$b^I$} (Ix)
	(Py) edge[->] node[above] {$b^P$} (Px);
\end{tikzpicture}}%
%
\subcaptionbox*{
	Direct target support \\ for child nodes \\ ($\istdir b{x,a}$)
}[.5\textwidth]{
	\begin{tikzpicture}[on grid,baseline=(Iv)]
	\node (Iv) {};
	\node (Pv) [below=of Iv] {};
	\node (Ix) [below=of Pv] {};
	\node (Px) [below=of Ix] {};
	\node (Prt) [above=of Iv] {};
	
	\path
	(Iv) edge[->] node[opengraph] {} node[left] {$d_v$} (Pv)
	(Iv) edge[->] node[left] {$u_v$} (Prt);
	
	\path[allcolor=gray]
	(Ix) edge[->] node[opengraph] {} node[left] {$d_x$} (Px)
	(Ix) edge[->] node[left] {$u_x$} (Pv);
	
	\node (Iw) [right=3 of Iv] {};
	\node (Pw) [below=of Iw] {};
	
	\path
	(Iw) edge[->] node[opengraph] {} node[right] {$d_w$} (Pw);
	
	\path[morphism]
	(Iv) edge[->] node[above] {$a^I$} (Iw)
	(Pv) edge[->] node[above] {$a^P$} (Pw);
	
	\node (Iy) [below=of Pw] {};
	\node (Py) [below=of Iy] {};
	
	\path
	(Iy) edge[->] node[opengraph] {} node[right] {$d_y$} (Py)
	(Iy) edge[->] node[right] {$u_y$} (Pw);
	
	\path[morphism=blue!40!white]
	(Iy) edge[->] node[above] {$b^I$} (Ix)
	(Py) edge[->] node[above] {$b^P$} (Px);
\end{tikzpicture}
}%

	\caption{The notions of support of \cref{def:target support}}
	\label{fig:target-support}
\end{figure}

\begin{definition}[syntactic preservation]
Let $\cM\of \cT\to\cU$ be an CT-morphism, and let $\trip vaw\in\cM$.
\begin{itemize}[topsep=\smallskipamount]
\item \emph{(Direct preservation.)} $a:\pdir$ if for every child $y$ of $w$, there is some $b:\tdir{y,a}$ such that $b:\pdir$.
\item \emph{(Child-based preservation.)} $a:\pchd$ if one of the following holds:
\begin{itemize}
\item for every child $y$ of $w$, there is some $b:\tdir{y,a}$ such that $b:\pchd$;
\item there is a child $x$ of $v$ with an open graph morphism $c\of d_x\to d_v$, such that for every child $y$ of $x$ there is some $\trip ybw\in\cM$ with $b:\pchd$.
\end{itemize}
\end{itemize}
\end{definition}

