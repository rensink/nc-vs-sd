\section{Definitions}
\label{sec:definitions}

\begin{itemize}
\item An \emph{interface} $I$ is a discrete graph.

\item An \emph{open graph} is an arrow $g\of I\to G$ where $I$ is an interface; we say that $g$ has \emph{interface} $I_g=I$ and \emph{pattern} $P_g=G$. Open graphs are used for different purposes: within a condition to associate (both upper and lower) interfaces with the patterns, and also as assignments --- that is, the objects on which satisfaction is defined.

\item An \emph{interface morphism} $f$ from $I$ to $J$ is a graph morphism $f:I\to J$.

\item An \emph{open graph morphism} $a\of g\to h$ a pair of graph morphisms $a\I:I_g\to I_h, a\P:P_g\to P_h$ such that $a\I;h=g;a\P$.
\end{itemize}
%
\begin{definition}[condition tree]\label{def:condition tree}
A \emph{condition tree} $\cT$ is a quadruple of the form $\tupof{V, {\parentof}, \setof{d_v}_{v\in V}, \setof{u_v}_{v\in V}}$ in which

\begin{itemize}[topsep=\itemsep]
\item $V$ is a finite, non-empty set of vertices;

\item ${\parentof}\subseteq V\times V$ is a parent-child relation, such that one $v\in P$ (the \emph{root}, denoted $\rt$) has no predecessor, and all other elements of $V$ have exactly one predecessor (their \emph{parent});

\item for all $v\in V$, $d_v$ is an open graph. We denote $I_v=I_{d_v}$ and $P_v=P_{d_v}$.

\item for all $v\parentof w$, $u_w$ is a graph morphism $I_w\to P_v$, whereas $u_{\rt}=\id_{I_{\rt}}$.
\end{itemize}
\end{definition}
%
We use $\ancestof$ and $\ancesteq$ to denote the transitive, respectively transitive and reflexive, closure of $\parentof$. The root of a condition tree $\cT$ is denoted $\rtof\cT$; we denote $I_\cT=I_{\rtof\cT}$, $P_\cT=P_{\rtof\cT}$ and $d_\cT=d_{\rtof\cT}$. For an arbitrary vertex $v\in V_\cT$, $\hat v$ will denote the subtree rooted at $v$. We also use a \emph{sibling} relation, ${\sibling}\subseteq V\times V$, as the smallest equivalence over $V$ such that $v\childof\parentof w$ implies $v\sibling w$; i.e., $v\sibling w$ if either $v$ and $w$ are (possibly identical) children of the same parent, or if $v=w=\rt$.\footnote{In fact, this use of the word \emph{sibling} deviates from its meaning in the natural world since there it is not a reflexive relation; however, we use it for lack of a good alternative (\emph{co-child} would be cumbersome).} For an arbitrary vertex $v\in V_\cT$ we use $U_v$ to denote the codomain of $u_v$; hence $U_w=P_v$ if $w\childof v$ and $U_w=I_\cT$ if $w=\rtof\cT$.
Finally, for an arbitrary vertex $v\in V_\cT$ we denote by $\depth(\hat v)$ the \emph{depth} of the corresponding subtree, defined as $\depth(\hat v) = 0$ if $v$ has no children in $\cT$, and $\depth(\hat v) = max \{\depth(\hat u) \mid v \parentof u\} + 1$.
\todo{This is new notation; please comment}

\begin{definition}[satisfaction]\label{def:satisfaction}
Let $\cT$ be a condition tree and $g\of I_\cT\to G$ a graph morphism. $g$ \emph{satisfies $\cT$} denoted $g\sat \cT$, if there is a graph morphism $h\of P_\cT\to G$ such that
\begin{inumerate}
\item $g=d_\cT;h$, and 
\item $u_v;h\nsat \hat v$ for all children $v$ of $\rtof\cT$.
\end{inumerate}
$h$ is then called a \emph{witness} of $g\sat \cT$.
\end{definition}
%
If $g$ satisfies $\cT$, we also call $g$ a $\cT$-model.

\todo[inline]{At this point introduce the connection from condition trees to FOL}

\begin{definition}[morphism]\label{def:morphism}
Given two condition trees $\cT,\cU$ with $I_\cT=I_\cU=I$, a morphism $\cM=\tupof{\tp,\rt}$ from $\cT$ to $\cU$ is a tuple consisting of
\begin{itemize}
\item a set $\tp$ of triples $(v,a,w)$ where $(v,w)\in (V_\cT\times V_\cU) \cup (V_\cU\times V_\cT)$ and $a\of d_v\to d_w$ is an open graph morphism;
\item a distinguished element $\rt\in\tp$, called the \emph{root}, such that $\rt=(\rtof\cT,a_{\rt},\rtof\cU)$ with $a_{\rt}\I=\id_I$.
\end{itemize}
\end{definition}
%
We typically denote a triple $(v,a,w)$ of the kind above as $\trip vaw$; moreover, we usually associate $\cM$ with its component $\tp$ and write $\trip vaw\in\cM$, using $\rtof\cM$ to denote the root of $\cM$. Moreover, we also sometimes use the component $a$ (with constituents $a\I$ and $a\P$) to refer to the entire triple $\trip vaw$, when that causes no confusion and the identity of $v$ and $w$ is clear from the context. \Cref{fig:triple} visualises the components of a triple $\trip vaw$.

\begin{figure}
\centering
\begin{tikzpicture}[on grid]
\node (Iv) {$I_v$};
\node (Pv) [below=of Iv] {$P_v$};

\path
  (Iv) edge[->] node[opengraph] (v) {} node[left] {$d_v$} (Pv)
  (Iv) node {$I_v$}
  (Pv) node {$P_v$};


\node (Iw) [right=2 of Iv] {$I_w$};
\node (Pw) [below=of Iw] {$P_w$};

\path
  (Iw) edge[->] node[opengraph] (w) {} node[right] {$d_w$} (Pw)
  (Iw) node {$I_w$}
  (Pw) node {$P_w$};

\path[morphism]
  (v) edge[openmorphism] (w)
  (Iv) edge[->] node[above] {$a\I$} (Iw)
  (Pv) edge[->] node[above] {$a\P$} (Pw);
\end{tikzpicture}

\caption{Visualisation of a triple $\trip vaw$ as in \cref{def:morphism}. The light blue ellipses represent open graphs corresponding to nodes of $\cT$ and $\cU$, and the (fat) light blue arrow represents the triple as a whole. For the root triple, $I_v=I_w=I$ and $a\I=\id_I$}
\label{fig:triple}
\end{figure}
\begin{comment}
\todo[inline]{We could extend the child relation to triples, as follows.}
\noindent Given a morphism $\cM$, we define a parent relation ${\parentof}\subseteq \cM\times \cM$ as follows:
\[ (v,a,w)\parentof (y,b,x) \;\Leftrightarrow\; v\parentof x\wedge w\parentof y \wedge u_y=b\I;u_x;a\P \enspace. \]
\todo[inline]{This might provide a more acceresible and uniform way to define the notions of direct (preservation and reflection) evidence.}
\end{comment}
%
Given a condition tree $\cT$ with root $\rt$, the identity morphism $\cI_\cT\of \cT\to\cT$ is defined as
\[ \cI_\cT = \tupof{\gensetof{(v,\id_{d_v},v)}{v\in V_\cT},(\rt,\id_{d_{\rt}},\rt)} \enspace. \]
Given two condition tree morphisms $\cM\of \cT\to \cU,\cN\of \cU\to \cV$, their composition $\cM;\cN\of \cT\to \cV$ is defined as
\[ \cM;\cN =
\begin{array}[t]{@{}l}
	\langle\{ (v,a;b,x) \mid \exists w\st \left( (v,a,w) \in \cM \wedge (w,b,x)\in \cN\right) \vee\\ 
	\left( (v,a,w) \in \cN \wedge (w,b,x)\in \cM\right)\},(\rtof\cT,a_{\rtof\cM};a_{\rtof\cN},\rtof\cV)\rangle\enspace.
\end{array}
\]
