\section{Typing morphisms}
\label{sec:typing morphisms}


In the previous section after introducing the category $\NC$ of nested conditions we discussed several structural properties that morphisms of $\NC$ may satisfy. In this section we will consider several wide sub-categories of $\NC$, i.e.~categories having all the objects of $\NC$ but only some proper subsets of morphisms, and we will relate them by obvious functors. 

The properties of morphisms we have considered are structural, thus in a sense syntactical. But the notion of satisfaction of conditions naturally induces a relation of \emph{entailment} among conditions having the same root, that gives rise to a ``semantic'' category, $\NC^{\models}$, which is actually a preoder. 

\begin{definition}[semantic entailment and equivalence]
Given two conditions $\cT$ and $\cU$ having the same root,  we define \emph{semantic entailment} $\cT \entails \cU$ and \emph{semantic equivalence} $\cT \equiv \cU$ as follows:
\begin{align*}
\cT \entails \cU & \text{ if for all arrows $g$: } g\sat \cT \text{ implies } g\sat \cU \\
\cT \equiv \cU & \text{ if for all arrows $g$: } g\sat \cT \text{ if and only if } g\sat \cU \enspace. 
\end{align*}

Category $\NC^{\models}$ has the same objects of $\NC$, and for each pair of conditions $\cT,\cU$ one arrow from $\cT$ to $\cU$ iff $\cT \models \cU$. Since there is at most one arrow between two objects, $\NC^{\models}$ is a preorder.
\end{definition}

One main concern will be to relate the syntactic categories of conditions, based on structural morphisms, with the semantic category $\NC^{\models}$. 

In order to deal with the several kinds of morphisms introduced before, we will consider a typing system over arrows of $\NC$. For a property \emph{prop} on such arrows, we will write $(\cM \of \cT \to \cV) \of \mathit{prop}$ to state that arrow $\cM$ satisfies it, or simply $\cM \of \mathit{prop}$ if the source and the target conditions of $\cM$ are understood. We will summarize various results relating properties of arrows through inference rules, with the usual meaning: if the typing judgements of the premises are true then also the typing judgement in the consequence is provably true. Therefore every inference rule represents a proof obligation.

Let us start introducing the properties of interest.

\begin{definition}[properties of arrows of $\NC$]
\label{def:properties}
\end{definition}
Let $\cT$ and $\cU$ be conditions with the same root ($I_\cT=I_\cU$) and $\cM \of \cT \to \cU$ be an arrow of $\NC$. Then we will write:

\begin{description}
    \item[{[$\rdir$]}] $\cM \of \rdir$ if $\cM$ is source-saturated using only the  direct rule;
    \item [{[$\rchd$]}] $\cM \of \rchd$ if $\cM$ is source-saturated using the  direct and the child rules (but not the sibling rule);
    \item  [{[$\rsib$]}] $\cM \of \rsib$ if $\cM$ is source-saturated using the  direct and the sibling rules (but not the child rule);
    \item [{[$\pdir$]}] $\cM \of \pdir$ if $\cM$ is target-saturated using only the  direct rule;
    \item [{[$\pchd$]}] $\cM \of \pchd$ if $\cM$ is target-saturated using the  direct and the child rules;
    % \item [{[$\rfl$]}] $\cM \of \rfl$ if $\cM$ reflects satisfaction;
    % \item [{[$\prs$]}] $\cM \of \prs$ if $\cM$ preserves satisfactions;
\end{description}

Based on the results of the previous section, we can state the following facts.

\begin{proposition}[relations among properties of $\NC$ arrows]
The following implications, presented as inference rules, hold for all arrows $\cM \of\cT \to \cV$ and $\cN\of \cV \to \cU$: 

$$\infer[{[1]}]{\cI_\cT \of \rdir}{} \quad \infer[{[2]}]{\cM\of \rchd}{\cM\of \rdir} \quad \infer[{[3]}]{\cM\of \rsib}{\cM\of \rdir} \quad \infer[{[4]}]{\cM;\cN\of \rchd}{\cM\of \rchd & \cN \of \rchd}  \quad \infer[{[5]}]{\cM;\cN\of \rsib}{\cM\of \rsib & \cN \of \rsib} $$


$$\infer[{[9]}]{\cI_\cT \of \pdir}{} \quad \infer[{[10]}]{\cM\of \pchd}{\cM\of \pdir} \quad 
% \infer[{[1]}]{\cM\of \rsib}{\cM\of \rdir} \quad 
\infer[{[11]}]{\cM;\cN\of \pchd}{\cM\of \pchd & \cN \of \pchd}  $$
\end{proposition} 

\begin{definition}[properties $\rfl$ and $\prs$]
    \label{def:prop-rfl-prs}
    The properties $\rfl$ (provably reflects satisfaction) and $\prs$ (provably preserves satisfaction) are defined by the following rules:

$$\infer[{[6]}]{\cM\of \rfl}{\cM\of \rchd} \quad \infer[{[7]}]{\cM\of \rfl}{\cM\of \rsib} \quad \infer[{[8]}]{\cM;\cN\of \rfl}{\cM\of \rfl & \cN \of \rfl} $$ 

$$\infer[{[12]}]{\cM\of \prs}{\cM\of \pchd} \quad
% \infer{\cM\of \rfl}{\cM\of \rsib} \quad 
\infer[{[13]}]{\cM;\cN\of \prs}{\cM\of \prs & \cN \of \prs} $$ 
\end{definition}


The properties just summarized allow us to define easily some wide subcategories of $\NC$.

\begin{definition}[subcategories of $\NC$]
    \begin{itemize}
    \item $\NC_\rchd$ includes all arrows $\cM$ such that $\cM\of \rchd$; 
    \item $\NC_\rsib$ includes all arrows $\cM$ such that $\cM\of \rsib$;
    \item $\NC_\rfl$ includes all arrows $\cM$ such that $\cM\of \rfl$;
    \item  $\NC_\pchd$ includes all arrows $\cM$ such that $\cM\of \pchd$;
    \item $\NC_\prs$ includes all arrows $\cM$ such that $\cM\of \prs$.
    \end{itemize}
    
    The above categories are well defined: in fact each class of arrows considered includes identities and is closed under composition. The unit and associativity laws are automatically satisfied because all arrows belong to category $\NC$.     
\end{definition}

\begin{proposition}[functors among categories of nested conditions]
    There are obvious inclusion functors 
    \begin{itemize}
    \item $\NC_\rchd \hookrightarrow \NC_\rfl$
    \item $\NC_\rsib \hookrightarrow \NC_\rfl$
    \item $\NC_\pchd \hookrightarrow \NC_\prs$
    \item $\NC_\rfl \hookrightarrow \NC$
    \item $\NC_\prs \hookrightarrow \NC$
    \end{itemize}
    Furthermore, the following functors relate syntactic categories of structural morphisms among nested conditions with the preorder of entailment among conditions:
    \begin{itemize}
    \item $\NC_\rfl^{op} \to \NC^{\models}$
    \item $\NC_\prs \to \NC^{\models}$
    \end{itemize} 
\end{proposition}

The last two functors state a form of soundness: (1) if there is an $\rfl$-arrow from $\cU$ to $\cT$ , then $\cT \entails \cU$, and (2)  if there is a $\prs$-arrow from $\cT$ to $\cU$, then $\cT \entails \cU$. 
The main goal of the next section is enrich this ``soundness" result to achieve ``completeness''. That is, we introduce further syntactic categories of nested conditions such that there is an arrow between two conditions IF AND ONLY IF the second entails the first one.

\subsection{Path categories of nested conditions}

\begin{definition}[path category of nested conditions]
    Let $\NC^*$ be the category having all the objects of $\NC$, and undirected paths of arrows of $\NC$ as arrows, denoted $\cP\of \cT \leadsto \cU$. 

    More explicitly, undirected paths between objects are defined by the following rules, where if $\cM\of \cT \to \cU$ is an arrow of $\NC$ by $\op{\cM}: \cU \leadsto \cT$ we denote the path of length one going the other way around.

    $$ \infer{\cI_\cT\of \cT \leadsto \cT}{\cI_\cT\of \cT \to \cT} \qquad 
    \infer{\cP \cdot \cM \of \cT \leadsto \cV}{\cP\of \cT \leadsto \cU & \cM \of \cU \to \cV} \qquad 
    \infer{\cP \cdot \op{\cM} \of \cT \leadsto \cV}{\cP\of \cT \leadsto \cU & \cM \of \cV \to \cU}$$

    Alternatively:\todo{AC: I think this is better, as it fits better with the equations. Should }

    $$ \infer{\cI_\cT\of \cT \leadsto \cT}{\cI_\cT\of \cT \to \cT} \qquad 
    \infer{\cM \of \cT \leadsto \cU}{\cM \of \cT \to \cU} \qquad 
    \infer{\op{\cM} \of \cU \leadsto \cT}{\cM \of \cT \to \cU}\qquad
    \infer{\cP \cdot \cQ \of \cT \leadsto \cV}{\cP\of \cT \leadsto \cU & \cQ \of \cU \leadsto \cV}$$

    Paths are subject to the obvious equations making $\NC^*$ a category:

    $$ \cI_\cT \cdot \cP = \cP = \cP \cdot \cI_\cU \mbox{\quad if\quad} \cP\of \cT \leadsto \cU$$
    $$\cP \cdot (\cQ \cdot \cR) = (\cP \cdot \cQ) \cdot \cR$$
    
    The $\op{\_}$ operation extends to paths in the obvious way\todo{AC: there should be a general name for this kind of operation}, so that if $\cP\of \cT \leadsto \cU$ then $\op{\cP}\of \cU \leadsto \cT$:
   $$ \op{\cI}_\cT = \cI_\cT \qquad \op{(\cP\cdot \cM)} = \op{\cM}\cdot \op{\cP} \qquad \op{(\cP\cdot \op{\cM})} = \cM\cdot \op{\cP} $$ 
   Alternatively:
$$ \op{\cI}_\cT = \cI_\cT \qquad 
\op{(\op{\cM})} = \cM \qquad \op{(\cP\cdot \cQ)} = \op{\cQ} \cdot \op{\cP} $$ 
\end{definition}

We extend now the properties of Definition \ref{def:prop-rfl-prs} to arrows of $\NC^*$.

\begin{definition}[properties of paths]
    $$\infer{(\cM \of \cT \leadsto \cU)\of \prs}{(\cM \of \cT \to \cU)\of \prs}\qquad \infer{(\cM \of \cT \leadsto \cU)\of \rfl}{(\cM \of \cT \to \cU)\of \rfl}\qquad \infer{(\op{\cM} \of \cU \leadsto \cT)\of \prs}{(\cM \of \cT \to \cU)\of \rfl}
    \qquad \infer{(\op{\cM} \of \cU \leadsto \cT)\of \rfl}{(\cM \of \cT \to \cU)\of \prs}$$
$$\infer{\cP\cdot\cQ\of \prs}{\cP\of \prs & \cQ \of \prs} \qquad \infer{\cP\cdot\cQ\of \rfl}{\cP\of \rfl & \cQ \of \rfl} $$

Furhter, let $\NC^*_{\prs}$ be the wide subcategory of $\NC^*$ containing as arrows all the paths $\cP$ such that  $\cP \of \prs$, and similarly $\NC^*_{\rfl}$ be the one containing all paths $\cP$ such that  $\cP \of \rfl$.

\end{definition}

\begin{theorem}[soundness and completeness]
    There are identity-on-objects functors  $\op{(\NC^*_{\rfl})} \to \NC^{\entails}$ and $\NC^*_{\prs} \to \NC^{\entails}$. Furthermore, both functors are sujective on arrows. 

    That is, for conditions $\cT$, $\cU$ having the same root, if $\cT \entails \cU$ then there is an undirected path $\cP:\cU \leadsto \cT $satisfying property $\rfl$, \dots
\end{theorem}



