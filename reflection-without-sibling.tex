\section{Model reflection}
\label{sec:reflection}

With the above in mind, we first define reflection of models, this being the more straightforward of the two directions.

\begin{definition}[model reflection]
Let $\cM\of \cT\to\cU$ be a morphism.
\begin{itemize}[topsep=\itemsep]
\item $\trip vaw\in\cM$ \emph{reflects a $\hat w$-model $g$} if $a\I$ is a $\hat v$-model. We say that $\trip vaw$ \emph{reflects models} if it reflects all $\hat w$-models.

\item $\cM$ \emph{reflects models} if $\rtof\cM$ reflects models.
\end{itemize}
\end{definition}
%
Note that, in case $\trip vaw=\rtof\cM$, we have $a\I=\id_I$; hence the condition of model reflection for morphisms implies that, if $g$ is a model of $\cU=\hat w$, then $g=a\I;g$ is a model of $\hat v=\cT$; in other words, the models of $\cU$ form a subset of those of $\cT$.
%
A given triple $\trip vaw$ can be shown to reflect models if the morphism as a whole provides enough internal \emph{evidence} for reflection. In particular, for a candidate model-reflecting triple $\trip vaw\in \cM$, for every child $x$ of $v$ we require the existence of further triples in $\cM$ (the evidence). To obtain a rich enough framework, we introduce diverse forms of evidence.

Given a morphism $\cM$ and a triple $\trip vaw\in\cM$, we say that a triple $\trip ybx\in\cM$ is an \emph{$\sX$-child} of $\trip vaw$, denoted %$(\trip vaw)\xchild (\trip ybx)$ or 
$(v,y) \xpattern ab (w,x)$, if they establish the first pattern in \Cref{fig:child patterns}, and that a triple $\trip ybw\in \cM$ is a \emph{$\sV$-child under $k$} of $\trip vaw$, denoted %$(v,a,w)\vchild[k] (y,b,w)$ or 
$(v,y) \vpattern[k] ab w$, if they establish the second pattern in \Cref{fig:child patterns}. In the latter, case, we often omit the $k$ and just write $(v,y) \vpattern ab w$ to mean that there is a $k$ such that $(v,y) \vpattern[k] ab w$.
%
\begin{figure}
  \centering
  \subcaptionbox*{
  $\sX$-child relation: $(v,x)\xpattern ab (w,y)$
}[.5\textwidth]{
\begin{tikzpicture}[on grid,baseline=(Iv),inner sep=1]
\node (Iv) {};
\node (Pv) [below=of Iv] {};
\node (Ix) [below=of Pv] {};
\node (Px) [below=of Ix] {};

\path
  (Iv) edge[draw=weakColor,->]
       node[opengraph] (v) {}
       node[left,weakColor] {$d_v$} (Pv)
%  (Ix) edge[->] node[opengraph] {} node[left] {$d_x$} (Px)
  (Ix) edge[draw=weakColor,->]
       node[opengraph] (x) {}
       node[left,weakColor] {$d_x$} (Px)
  (Ix) edge[->] node[left] {$u_x$} (Pv);
  
\node (Iw) [right=2.5 of Iv] {};
\node (Pw) [below=of Iw] {};

\path
  (Iw) edge[weakColor,->] 
       node[opengraph] (w) {} 
       node[right,weakColor] {$d_w$} (Pw);

\path[morphism]
  (v) edge[openmorphism] (w)
  (Iv) edge[morphism=morphismWeakColor,->] 
       node[morphismWeakColor,above] {$a\I$} (Iw)
  (Pv) edge[->] node[above] {$a\P$} (Pw);

\node (Iy) [below=of Pw] {};
\node (Py) [below=of Iy] {};

\path
%  (Iy) edge[->] node[opengraph] {} node[right] {$d_y$} (Py)
  (Iy) edge[draw=weakColor,->] 
       node[opengraph] (y) {} 
       node[inner sep=1,right,weakColor] {$d_y$} (Py)
  (Iy) edge[->] node[right] {$u_y$} (Pw);

\path[morphism]
  (y) edge[openmorphism] (x)
  (Iy) edge[->] node[above] {$b\I$} (Ix)
  (Py) edge[morphismWeakColor,->] 
       node[morphismWeakColor,above] {$b\P$} (Px)
  ;

% Defs to make sure bottom is aligned
\node (I) [below=1 of Px] {};
\node (P) [below=1 of I] {};
\path[draw=none]
  (I) to node[opengraph,fill=none] {} (P);
\end{tikzpicture}}%
%
\subcaptionbox*{
  $\sV$-child relation: $(v,y)\vpattern[k] ab w$
}[.5\textwidth]{
\begin{tikzpicture}[on grid,baseline=(Iv),inner sep=1]
\node (Iv) {};
\node (Pv) [below=of Iv] {};
\node (Ix) [below=of Pv] {};
\node (Px) [below=of Ix] {};
\node (Iy) [below=of Px] {};
\node (Py) [below=of Iy] {};

\path
  (Iv) edge[weakColor,->] 
       node[opengraph] (v) {} 
       node[weakColor,left] {$d_v$} (Pv)
  (Ix) edge[->] 
       node[opengraph] (x) {} 
       node[left] {$d_x$} (Px)
  (Ix) edge[->] node[left] {$u_x$} (Pv)
  (Iy) edge[->] 
       node[opengraph] (y) {} 
       node[left] {$d_y$} (Py)
  (Iy) edge[->]
       node[left] {$u_y$} (Px);
% The up-arrow
\path
    (Iy) edge[bend right,->] 
         node[above right,near end] {$k$} (Ix);
  
\node (Iw) [right=2.5 of Iv] {};
\node (Pw) [below=of Iw] {};

\path
  (Iw) edge[weakColor,->] 
       node[opengraph] (w) {} 
       node[weakColor,right] {$d_w$} (Pw);

% morphism b, drawn before a to get overlap right
\path[morphism]
  (y.53) edge[openmorphism,shorten <=-5pt] (w.239)
  (Iy.east) edge[->,morphismWeakColor] 
            node[left,morphismWeakColor] {$b\I$} (Iw)
  (Py.east) edge[->] 
            node[right] {$b\P$} (Pw)
  ;

% morphism a
\path[morphism]
  (v) edge[openmorphism] (w)
  (Iv) edge[->,morphismWeakColor] 
       node[above,morphismWeakColor] {$a\I$} (Iw)
  (Pv) edge[->]
       node[above] {$a\P$} (Pw)
  ;
\end{tikzpicture}
}%

  \caption{$\sX$- and $\sV$-child patterns; the bold black/blue subdiagrams commute}
  \label{fig:child patterns}
\end{figure}
%
These patterns satisfy the following (rather technical) properties.
%
\begin{lemma}\label{lem:evidence}
Let  $\cM\of \cT\to\cU$ be a condition tree morphism, let $\trip vaw\in\cM$ and $x\childof v$. Let $g\sat \hat w$ with witness $h$.
\begin{enumerate}[topsep=\itemsep]
\item\label{re-dir} If $(v,x) \xpattern ab (w,y)$ and $\trip ybx$ reflects models, then $u_x;a\P;h\nsat \hat x$.
\item\label{re-chd} If $(v,y) \vpattern ab w$ for some $y\childof x$ and $\trip ybw$ reflects models, then $u_x;a\P;h\nsat \hat x$.
\end{enumerate}
\end{lemma}
%
\begin{proof}
By contradiction. In each case, assume $u_x;a\P;h\sat \hat x$.
\begin{enumerate}
\item $(v,x) \xpattern ab (w,y)$ means that $w\parentof y$ and $b\I;u_x;a\P=u_y$. Since $b$ reflects models, $u_x;a\P;h\sat \hat x$ implies $b\I;u_x;a\P;h\sat \hat y$, implying (due to $u_y;h=b\I;u_x;a\P;h$) that $u_y;h\sat \hat y$, which contradicts $g\sat \hat w$.

\item $(v,y) \vpattern ab w$ with $y\childof x$ means there is a mediating morphism $k\of I_y\to I_x$ such that $k;d_x=u_y$ and $k;u_x;a\P=d_y;b\P$. Since $u_x;a\P;h\sat \hat x$, there must be some witness $f\of P_x\to G$ such that $u_x;a\P;h=d_x;f$ and $u_y;f\nsat \hat y$. Since $b$ reflects models, $b\I;g\sat \hat y$; since $b\I;g= b\I;d_w;h=d_y;b\P;h= k;u_x;a\P;h=k;d_x;f=u_y;f$, this gives rise to a contradiction.
\end{enumerate}
\end{proof}
%
Based on these notions of evidence, we define corresponding notions of \emph{syntactic reflection}:

\begin{definition}[syntactic reflection]\label{def:syntactic reflection}
Let $\cM\of \cT\to\cU$ be a condition tree morphism. $\rgen_\cM\subseteq \cM$ is the smallest set containing all $\trip vaw\in \cM$ such that for every child $x$ of $v$ one of the following holds:
\begin{enumerate}
\item either there is some $\trip ybx\in \rgen_\cM$ such that $(v,x)\xpattern ab (w,y)$;
\item or there are $y\childof x$ and $\trip ybw\in \rgen_\cM$ such that $(v,y)\vpattern ab w$.
\end{enumerate}
\end{definition}
%
We typically write $\rgen$ rather than $\rgen_\cM$ if $\cM$ is clear from the context. Note that, if $v$ does not have children, then $\trip vaw\in \rgen$ for all $\trip vaw\in\cM$. The latter observation will form the base case of our induction proofs. In fact, if we define $\rgen^0$ as the set of all $\trip vaw\in\cM$ such that $v$ does not have children, and $\rgen^i$ for $i>0$ as the set of all $\trip vaw\in\cM$ such that for all $x\childof v$ one of the clauses in \Cref{def:syntactic reflection} holds for some $b\in \rgen^{i-1}$, then clearly $\rgen=\bigcup_{i\geq 0} \rgen^i$.

\begin{lemma}[syntactic reflection implies model reflection]\label{lem:reflection}
If $\cM\of \cT\to \cU$ is a condition tree morphism, then all $a\in\rgen_\cM$ reflect models.
\end{lemma}
%
\begin{proof}
By induction on $i$ in $\rgen^i$.
%
Consider $\trip vaw\in \rgen$ and let $g\sat \hat w$; we set out to prove that $a\I;g\sat \hat v$. Let $h\of P_w\to G$ be the witness for $g\sat \hat w$. It follows that $a\I;g=a\I;d_w;h=d_v;a\P;h$, hence $a\P;h$ is a prospective witness for $a\I;g\sat \hat v$.

Assume that $a\in \rgen^i$ and the property has been proved for all $a\in \rgen^j$ with $j<i$. If $i=0$, then $v$ has no children, meaning that there is nothing left to prove. Otherwise, let $x\childof v$ be arbitrary; we have to prove that $u_x;a\P;h\nsat \hat x$. By definition of $\rgen^i$, one of the following clauses applies.
\begin{enumerate}
\item there is some $\trip ybx\in \rgen^{i-1}$ such that $(v,x)\xpattern ab (w,y)$

\item there are $y\childof x$ and $\trip ybw\in \rgen^{i-1}$ such that $(v,y)\vpattern ab w$.
\end{enumerate}
In either case, by the induction hypothesis, $b$ reflects models, and hence \Cref{lem:evidence} implies that $u^x;a\P;h\nsat \hat x$.
\end{proof}

\begin{corollary}
If $\cM$ is a condition tree morphism, then $\rtof\cM\in \rgen_\cM$ implies that $\cM$ reflects models.
\end{corollary}
