\section{Model reflection}
\label{sec:reflection}

With the above in mind, we first define reflection of models, this being the more straightforward of the two directions.

\begin{definition}[model reflection]
Let $\cM\of \cT\to\cU$ be a morphism.
\begin{itemize}[topsep=\itemsep]
\item $\trip vaw\in\cM$ \emph{reflects a $\hat w$-model $g$} if $a\I;g$ is a $\hat v$-model. We say that $\trip vaw$ \emph{reflects models} if it reflects all $\hat w$-models.

\item $\cM$ \emph{reflects models} if $\rtof\cM$ reflects models.
\end{itemize}
\end{definition}
%
Note that, in case $\trip vaw=\rtof\cM$, we have $a\I=\id_I$; hence the property of model reflection for morphisms implies that, if $g$ is a model of $\cU=\hat w$, then $g=a\I;g$ is a model of $\hat v=\cT$; in other words, the models of $\cU$ form a subset of those of $\cT$.

A given triple $\trip vaw$ can be shown to reflect models if the morphism as a whole provides enough internal \emph{evidence} for reflection. In particular, for a candidate model-reflecting triple $\trip vaw\in \cM$, for every child $x$ of $v$ we require the existence of further triples in $\cM$ (the evidence). To obtain a rich enough framework, we introduce diverse forms of evidence.

Given a morphism $\cM$ and a triple $\trip vaw\in\cM$, we say that a triple $\trip ybx\in\cM$ is an \emph{$\sX$-child} of $\trip vaw$, denoted %$(\trip vaw)\xchild (\trip ybx)$ or 
$(v,x) \xpattern ab (w,y)$, if they establish the first pattern in \Cref{fig:child patterns}, and that a triple $\trip ybw\in \cM$ is a \emph{$\sV$-child under $k$} of $\trip vaw$, denoted %$(v,a,w)\vchild[k] (y,b,w)$ or 
$(v,y) \vpattern[k] ab w$, if they establish the second pattern in \Cref{fig:child patterns}. In the latter, case, we often omit the $k$ and just write $(v,y) \vpattern ab w$ to mean that there is a $k$ such that $(v,y) \vpattern[k] ab w$.
%
\begin{figure}
  \centering
  \subcaptionbox*{
  $\sX$-child relation: $(v,x)\xpattern ab (w,y)$
}[.5\textwidth]{
\begin{tikzpicture}[on grid,baseline=(Iv),inner sep=1]
\node (Iv) {};
\node (Pv) [below=of Iv] {};
\node (Ix) [below=of Pv] {};
\node (Px) [below=of Ix] {};

\path
  (Iv) edge[draw=weakColor,->]
       node[opengraph] (v) {}
       node[left,weakColor] {$d_v$} (Pv)
%  (Ix) edge[->] node[opengraph] {} node[left] {$d_x$} (Px)
  (Ix) edge[draw=weakColor,->]
       node[opengraph] (x) {}
       node[left,weakColor] {$d_x$} (Px)
  (Ix) edge[->] node[left] {$u_x$} (Pv);
  
\node (Iw) [right=2.5 of Iv] {};
\node (Pw) [below=of Iw] {};

\path
  (Iw) edge[weakColor,->] 
       node[opengraph] (w) {} 
       node[right,weakColor] {$d_w$} (Pw);

\path[morphism]
  (v) edge[openmorphism] (w)
  (Iv) edge[morphism=morphismWeakColor,->] 
       node[morphismWeakColor,above] {$a\I$} (Iw)
  (Pv) edge[->] node[above] {$a\P$} (Pw);

\node (Iy) [below=of Pw] {};
\node (Py) [below=of Iy] {};

\path
%  (Iy) edge[->] node[opengraph] {} node[right] {$d_y$} (Py)
  (Iy) edge[draw=weakColor,->] 
       node[opengraph] (y) {} 
       node[inner sep=1,right,weakColor] {$d_y$} (Py)
  (Iy) edge[->] node[right] {$u_y$} (Pw);

\path[morphism]
  (y) edge[openmorphism] (x)
  (Iy) edge[->] node[above] {$b\I$} (Ix)
  (Py) edge[morphismWeakColor,->] 
       node[morphismWeakColor,above] {$b\P$} (Px)
  ;

% Defs to make sure bottom is aligned
\node (I) [below=1 of Px] {};
\node (P) [below=1 of I] {};
\path[draw=none]
  (I) to node[opengraph,fill=none] {} (P);
\end{tikzpicture}}%
%
\subcaptionbox*{
  $\sV$-child relation: $(v,y)\vpattern[k] ab w$
}[.5\textwidth]{
\begin{tikzpicture}[on grid,baseline=(Iv),inner sep=1]
\node (Iv) {};
\node (Pv) [below=of Iv] {};
\node (Ix) [below=of Pv] {};
\node (Px) [below=of Ix] {};
\node (Iy) [below=of Px] {};
\node (Py) [below=of Iy] {};

\path
  (Iv) edge[weakColor,->] 
       node[opengraph] (v) {} 
       node[weakColor,left] {$d_v$} (Pv)
  (Ix) edge[->] 
       node[opengraph] (x) {} 
       node[left] {$d_x$} (Px)
  (Ix) edge[->] node[left] {$u_x$} (Pv)
  (Iy) edge[->] 
       node[opengraph] (y) {} 
       node[left] {$d_y$} (Py)
  (Iy) edge[->]
       node[left] {$u_y$} (Px);
% The up-arrow
\path
    (Iy) edge[bend right,->] 
         node[above right,near end] {$k$} (Ix);
  
\node (Iw) [right=2.5 of Iv] {};
\node (Pw) [below=of Iw] {};

\path
  (Iw) edge[weakColor,->] 
       node[opengraph] (w) {} 
       node[weakColor,right] {$d_w$} (Pw);

% morphism b, drawn before a to get overlap right
\path[morphism]
  (y.53) edge[openmorphism,shorten <=-5pt] (w.239)
  (Iy.east) edge[->,morphismWeakColor] 
            node[left,morphismWeakColor] {$b\I$} (Iw)
  (Py.east) edge[->] 
            node[right] {$b\P$} (Pw)
  ;

% morphism a
\path[morphism]
  (v) edge[openmorphism] (w)
  (Iv) edge[->,morphismWeakColor] 
       node[above,morphismWeakColor] {$a\I$} (Iw)
  (Pv) edge[->]
       node[above] {$a\P$} (Pw)
  ;
\end{tikzpicture}
}%

  \caption{$\sX$- and $\sV$-child patterns; the bold black/blue subdiagrams commute}
  \label{fig:child patterns}
\end{figure}
%
These patterns satisfy the following (rather technical) properties.
%
\begin{lemma}\label{lem:evidence}
Let  $\cM\of \cT\to\cU$ be a condition tree morphism, let $\trip vaw\in\cM$ and $x\childof v$. Let $g\sat \hat w$ with witness $h$.
\begin{enumerate}[topsep=\itemsep]
\item\label{re-dir} If $(v,x) \xpattern ab (w,y)$ and $\trip ybx$ reflects models, then $u_x;a\P;h\nsat \hat x$.
\item\label{re-chd} If $(v,y) \vpatternH ab w$ for some $y\childof x$ and $\trip ybw$ reflects models, then $u_x;a\P;h\nsat \hat x$.
\end{enumerate}
\end{lemma}
%
\begin{proof}
By contradiction. In each case, assume $u_x;a\P;h\sat \hat x$.
\begin{enumerate}
\item $(v,x) \xpattern ab (w,y)$ means that $w\parentof y$ and $b\I;u_x;a\P=u_y$. Since $b$ reflects models, $u_x;a\P;h\sat \hat x$ implies $b\I;u_x;a\P;h\sat \hat y$, implying (due to $u_y;h=b\I;u_x;a\P;h$) that $u_y;h\sat \hat y$, which contradicts $g\sat \hat w$.

\item $(v,y) \vpattern ab w$ with $y\childof x$ means there is a mediating morphism $k\of I_y\to I_x$ such that $k;d_x=u_y$ and $k;u_x;a\P=d_y;b\P$. Since $u_x;a\P;h\sat \hat x$, there must be some witness $f\of P_x\to G$ such that $u_x;a\P;h=d_x;f$ and $u_y;f\nsat \hat y$. Since $b$ reflects models, $b\I;g\sat \hat y$; since $b\I;g= b\I;d_w;h=d_y;b\P;h= k;u_x;a\P;h=k;d_x;f=u_y;f$, this gives rise to a contradiction.
\end{enumerate}
\end{proof}
%
Based on these notions of evidence, we define corresponding notions of \emph{syntactic reflection}:

\begin{definition}[syntactic reflection]\label{def:syntactic reflection}
Let $\cM\of \cT\to\cU$ be a condition tree morphism. $\rgen_\cM\subseteq \cM$ is the smallest set containing all $\trip vaw\in \cM$ such that for every child $x$ of $v$ one of the following holds:
\begin{enumerate}
\item\label{xpattern} either there is some $\trip ybx\in \rgen_\cM$ such that $(v,x)\xpattern ab (w,y)$;
\item\label{vpattern} or there are $y\childof x$ and $\trip ybw\in \rgen_\cM$ such that $(v,y)\vpattern ab w$.
\end{enumerate}
\end{definition}
%
We typically write $\rgen$ rather than $\rgen_\cM$ if $\cM$ is clear from the context. The following property gives some insight in the existence of (alternative) syntactical reflections, in case the interface of the source node of a triple can be mapped in more than one way to the interface of the target node.\todo{A: Could alternatively (slightly more complicatedly) be formulated by defining an equivalence over triples and stating that $\rgen$ is well-defined modulo that equivalence.}
%
\begin{lemma}\label{lem:interface flexibility}
Let $\cM\of\cT\to \cU$ be a condition tree morphism and $\trip vaw\in \rgen_\cM$. If $\trip v{\bar a}w\in \cM$ such that $\bar a\P=a\P$ and $\bar a\I;d_w=a\I;d_w$, then $\trip v{\bar a}w\in \rgen_\cM$.
\end{lemma}
%
\begin{proof}
Immediate from the fact that neither of the clauses in \Cref{def:syntactic reflection} puts further constraints on $a\I$, beyond the fact that $a$ is an open graph morphism and hence $a\I;d_w=d_v;a\P$; and the latter equation is also satisfied by $\bar a$.
\end{proof}
%
As a basis for induction, we take the \emph{cumulative depth} of triples, defined as the sum of the depths of their source and target nodes (in the respective condition trees):
%
\[  \depth(\trip vaw) := \depth(\hat v)+\depth(\hat w) \enspace. \]
%
Note that $(v,x)\xpattern ab (w,y)$ implies $\depth(\trip ybx)< \depth(\trip vaw)$ and $(v,y)\xpattern ab w$ implies $\depth(\trip ybw)<\depth(\trip vaw)$.
%Note that, if $v$ does not have children, then $\trip vaw\in \rgen$ for all $\trip vaw\in\cM$. The latter observation will form the base case of our induction proofs. In fact, we define $\rgen^0$ as the set of all $\trip vaw\in\cM$ such that $v$ does not have children, and $\rgen^i$ for $i>0$ as the set of all $\trip vaw\in\cM$ such that for all $x\childof v$ one of the clauses in \Cref{def:syntactic reflection} holds for some $b\in \rgen^{i-1}$; then clearly $\rgen^i\subseteq \rgen^j$ for all $i\leq j$, and $\rgen=\bigcup_{i\geq 0} \rgen^i$.

\begin{proposition}[syntactic reflection implies model reflection]\label{lem:reflection}
If $\cM\of \cT\to \cU$ is a condition tree morphism, then all $\trip vaw\in\rgen_\cM$ reflect models.
\end{proposition}
%
\begin{proof}
By induction on $\depth(\trip vaw)$.
%
Consider $\trip vaw\in \rgen_\cM$ and let $g\sat \hat w$; we set out to prove that $a\I;g\sat \hat v$. Let $h\of P_w\to G$ be the witness for $g\sat \hat w$. It follows that $a\I;g=a\I;d_w;h=d_v;a\P;h$, hence $a\P;h$ is a prospective witness for $a\I;g\sat \hat v$.

If $\depth(\trip vaw)=0$, then $v$ has no children, meaning that there is nothing left to prove. Otherwise assume that $\depth(\trip vaw)=i>0$ and the proposition has been proved for triples of depth smaller than $i$. Let $x\childof v$ be arbitrary; we have to prove that $u_x;a\P;h\nsat \hat x$. 
By definition of $\rgen_\cM$, one of the following clauses applies.
\begin{enumerate}
\item there is some $\trip ybx\in \rgen_\cM$ such that $(v,x)\xpattern ab (w,y)$, where $\depth(\trip ybx)<i$.

\item there are $y\childof x$ and $\trip ybw\in \rgen_\cM$ such that $(v,y)\vpattern ab w$, where $\depth(\trip ybw)<i$.
\end{enumerate}


% By definition of $\rgen^i$, one of the following clauses applies.
% \begin{enumerate}
% \item there is some $\trip ybx\in \rgen^{i-1}$ such that $(v,x)\xpattern ab (w,y)$. It follows that $\depth(\trip ybx)<i$.

% \item there are $y\childof x$ and $\trip ybw\in \rgen^{i-1}$ such that $(v,y)\vpattern ab w$. It follows that $\depth(\trip ybw)<i$.
% \end{enumerate}
In either case, by the induction hypothesis, $b$ reflects models, and hence \Cref{lem:evidence} implies that $u^x;a\P;h\nsat \hat x$.
\end{proof}

\begin{corollary}
If $\cM$ is a condition tree morphism, then $\rtof\cM\in \rgen_\cM$ implies that $\cM$ reflects models.
\end{corollary}
%
An important further property is that syntactic model reflection composes, in the following sense.
%
\begin{proposition}\label{prop:syntactic reflection composes}
If $\cM:\cT\to\cU,\cN:\cU\to \cV$ be condition tree morphisms, then
\begin{itemize}
\item $\trip vaw\in \rgen_\cM$ and $\trip wbx\in \rgen_\cN$ implies $\trip v{a;b}x\in \rgen_{\cM;\cN}$;
\item $\trip vaw\in \rgen_\cN$ and $\trip wbx\in \rgen_\cM$ implies $\trip v{a;b}x\in \rgen_{\cM;\cN}$.
\end{itemize}
\end{proposition}
%
Part of the proof is based on the following, very technical, auxiliary lemma.
%
\begin{lemma}\label{lem:composition induction}
Let $\cM:\cT\to\cU$ and $\cN:\cU\to \cV$ be condition tree morphisms, let $\cX=\cM$ and $\cZ=\cN$ or vice versa, and let $\trip{y_1}{b_1}{z_1}\in \rgen_\cZ$ and $\trip{y_2}{a_2}{x_2}\in \rgen_\cX$ where $y_2\childof_\cU y_1$ (see \eqref{eq:composition induction premisse}). One of the following cases holds:
\begin{enumerate}
\item {[Left diagram of \eqref{eq:composition induction result}]} There is a child $z_2\childof z_1$ and a node $v\desceq_\cU y_2$ with $\trip {z_2}dy\in\rgen_\cZ$ and $\trip va{x_2}\in \rgen_\cX$, and an arrow $s:I_v\to I_{y_2}$ satisfying $\inlab{A1} s;a\I_2=c\I$ and $\inlab{A2} d\I;s;u_{y_2};b\P_1=u_{z_2}$.
% \item There is a child $z_2\childof z_1$ with $\trip {z_2}dy\in\rgen_\cZ$ and $\trip va{x_2}\in \rgen_\cX$ for some $v\desceq_\cU y_2$, and an arrow $s:I_v\to I_{y_2}$ satisfying $s;a\I_2=c\I$ and $d\I;s;u_{y_2};b\P_1=u_{z_2}$.

\item {[Right diagram of \eqref{eq:composition induction result}]}  There is a child $x_3\childof x_2$, a node $v\desceq_\cU y_2$ with an arrow $s:I_v\to I_{y_2}$, and triples $\trip{x_3}cv\in \rgen_\cX$ and $\trip vd{z_1}\in \rgen_\cZ$ satisfying $\inlab{B1} c\I;s;a\I_2;d_{x_2}=u_{x_3}$ and $\inlab{B2} s;u_{y_2};b\P_1=d_v;d\P$.
\end{enumerate}
% \item There is a child $x_3\childof x_2$ with $\trip{x_2}cv\in \rgen_\cX$ and $\trip vd{z_1}\in \rgen_\cZ$ for some $v\desceq_\cU y_2$, and an arrow $s:I_v\to I_{y_2}$ satisfying $c\I;s;a\I_2;d_{x_2}=u_{x_3}$ and $s;u_{y_2};b\P_1=d_v;d\P$.
% \end{enumerate}
\end{lemma}
%
\begin{proof}
Let us start with a visualisation of the statement of the lemma. Given a diagram of the following shape, where $b_1\in \rgen_\cZ$ and $a_2\in \rgen_\cX$:
%
\def\mypattern{
\node (Iy1) {};
\node (Py1) [below=of Iy1] {};
\node (Iy2) [below=of Py1] {};
\node (Py2) [below=of Iy2] {};
\path
  (Iy1) edge[maybeWeak,->] 
        node[opengraph] (y1) {} 
        node[maybeWeak,left] {$d_{y_1}$} (Py1)
  (Iy2) edge[draw=none] 
        node[opengraph] (y2) {} 
        (Py2)
  (Iy2) edge[->] node[left] {$u_{y_2}$} (Py1)
  ;
%
\node (Iz1) [right=2 of Iy1] {};
\node (Pz1) [below=of Iz1] {};
\path
  (Iz1) edge[maybeWeak,->]
        node[maybeWeak,opengraph] (z1) {} 
        node[right] {$d_{z_1}$} (Pz1)
  ;
%
\node (Ix2) [left=2 of Iy2] {};
\node (Px2) [below=of Ix2] {};
\path
  (Ix2) edge[maybeWeak2,->] 
        node[opengraph] (x2) {} 
        node[left] {$d_{x_2}$} (Px2)
  ;
%
\path[morphism]
  (y1) edge[openmorphism] (z1)
  (Iy1) edge[maybeWeakMorphism,->] 
        node[pos=.4,above] {$b\I_1$} (Iz1)
  (Py1) edge[->] node[pos=.4,above] {$b\P_1$} (Pz1)
  (y2) edge[openmorphism] (x2)
  (Iy2) edge[->] 
        node[pos=.4,above] {$a\I_2$} (Ix2)
  (Py2) edge[maybeWeakMorphism,->]
        node[pos=.4,above] {$a\P_2$} (Px2)
  ;

% interior of the y-nodes repainted no so as to come on top
\path
  (Iy2) edge[maybeWeak,->]
        node[left] {$d_{y_2}$} (Py2)
  ;
}
\colorlet{maybeWeak}{black}
\colorlet{maybeWeak2}{black}
\colorlet{maybeWeakMorphism}{morphismColor}
\begin{equation}\label{eq:composition induction premisse}
\begin{tikzpicture}[on grid,inner sep=1,baseline=(Py1)]
\mypattern
\end{tikzpicture}
\end{equation}
we need to prove that one of the following two completions exist, in which the subdiagrams formed by the bold black/blue arrows must commute:

\colorlet{maybeWeak}{weakColor}
\colorlet{maybeWeak2}{weakColor}
\colorlet{maybeWeakMorphism}{morphismWeakColor}
\begin{equation}\label{eq:composition induction result} 
\begin{tikzpicture}[on grid,inner sep=1,baseline=(Iy2)]
\mypattern
\node (Iz2) [below=of Pz1] {};
\node (Pz2) [below=of Iz2] {};
\node (Iv) [below=of Py2] {};
\node (Pv) [below=of Iv] {};
\path
  (Iz2) edge[weakColor,->]
        node[opengraph] (z2) {} 
        node[right] {$d_{z_2}$} (Pz2)
  (Iv) edge[draw=none]
       node[opengraph] (v) {}
       (Pv)
  (Iz2) edge[->] node[right] {$u_{z_2}$} (Pz1)
  ;
%
\path[morphism]
  (z2) edge[openmorphism] (v)
  (Iz2) edge[->]
        node[pos=.4,above] {$d\I$} (Iv)
  (Pz2) edge[morphismWeakColor,->] 
        node[pos=.3,above] {$d\P$} (Pv)
  (v) edge[openmorphism] (x2)
  (Iv) edge[->] 
       node[pos=.2,above right] {$c\I$} (Ix2)
  (Pv) edge[morphismWeakColor,->] 
       node[pos=.5,above right] {$c\P$} (Px2)
  ;

\path (Iv.north) edge[->,bend right=25] node[right] {$s$} (Iy2.south east);

% repaint an arrow
\path[morphism]
  (Py2) edge[morphismWeakColor,->] (Px2);
\path 
  (Iv) edge[weakColor,->]
       node[pos=.3,left] {$d_v$} 
       (Pv)
       ;
\end{tikzpicture}
\quad
\colorlet{maybeWeak2}{black}
\begin{tikzpicture}[on grid,inner sep=1,baseline=(Iy2)]
\mypattern
\node (Ix3) [below=of Px2] {};
\node (Px3) [below=of Ix3] {};
\node (Iv) [below=of Py2] {};
\node (Pv) [below=of Iv] {};
\path
  (Ix3) edge[weakColor,->]
        node[opengraph] (x3) {}
        node[left] {$d_{x_3}$} (Px3)
  (Iv) edge[draw=none] 
       node[opengraph] (v) {} 
       (Pv)
  (Ix3) edge[->] node[left] {$u_{x_3}$} (Px2)
  ;
%
\path[morphism]
  (v.60) edge[openmorphism] (z1)
  (Iv.east) edge[morphismWeakColor,->]
            node[pos=.4,above=.15] {$d\I$} (Iz1)
  (Pv.east) edge[->]
            node[above=.15] {$d\P$} (Pz1)
  (x3) edge[openmorphism] (v)
  (Ix3) edge[->]
        node[pos=.4,above] {$c\I$} (Iv)
  (Px3) edge[morphismWeakColor,->]
        node[pos=.4,above] {$c\P$} (Pv)
  ;

\path (Iv.north) 
      edge[->,bend right=25] 
      node[pos=.7,right] {$s$} (Iy2.south east);

% repaint an arrow
\path[morphism]
  (Py1) edge[->] (Pz1);
\path
  (Iv) edge[->]
       node[left] {$d_v$}
       (Pv);
\end{tikzpicture}
\end{equation}
%
The proof proceeds by induction on the depth of $\hat y_2$. \Cref{def:syntactic reflection} for $\trip{y_1}{b_1}{z_1}\in \rgen_\cZ$ and $y_2\childof y_1$ gives rise to the following cases, the second of which can only occur if $\depth(\hat y_2)>0$:
\begin{enumerate}
\item There is some $\trip{z_2}{b_2}{y_2}\in \rgen_\cZ$ such that $(y_1,y_2)\xpattern{b_1}{b_2} (z_1,z_2)$. By setting $v=y_2$, $c=a_2$, $d=b_2$ and $s=\id_{I_{y_2}}$ we have Clause~1 of this lemma, hence we are done.

\item There are $y_3\childof y_2$ and $\trip{y_3}{b_3}{z_1}\in \rgen_\cZ$ such that $(y_1,y_3)\vpattern[k]{b_1}{b_3} z_1$. This gives rise to the following diagram, where $\inlab{X1} k;d_{y_2} = u_{y_3}$ and $\inlab{X2} k;u_{y_2};b\P_1=d_{y_3};b\P_3$ hold: 
%
\def\mypattern{
\node (Iy1) {};
\node (Py1) [below=of Iy1] {};
\node (Iy2) [below=of Py1] {};
\node (Py2) [below=of Iy2] {};
\node (Iy3) [below=of Py2] {};
\node (Py3) [below=of Iy3] {};
\path
  (Iy1) edge[->] 
        node[opengraph] (y1) {} 
        node[left] {$d_{y_1}$} (Py1)
  (Iy2) edge[->] 
        node[opengraph] (y2) {} 
        (Py2)
  (Iy2) edge[->] node[left] {$u_{y_2}$} (Py1)
  (Iy3) edge[->] 
        node[opengraph] (y3) {} 
        node[left] {$d_{y_3}$} (Py3)
  (Iy3) edge[->] node[left] {$u_{y_3}$} (Py2)
  ;
%
\node (Iz1) [right=2.2 of Iy3] {};
\node (Pz1) [below=of Iz1] {};
\path
  (Iz1) edge[->] node[opengraph] (z1) {} node[right] {$d_{z_1}$} (Pz1)
  ;
%
\node (Ix2) [left=2.2 of Iy2] {};
\node (Px2) [below=of Ix2] {};
\path
  (Ix2) edge[->] node[opengraph] (x2) {} node[left] {$d_{x_2}$} (Px2)
  ;
%
\path[morphism]
  (y1.300) edge[openmorphism] (z1)
  (Iy1.east) edge[->] node[pos=.5,above=.2] {$b\I_1$} (Iz1.north west)
  (Py1.east) edge[->] node[pos=.4,above=.2] {$b\P_1$} (Pz1.north west)
  (y2) edge[openmorphism] (x2)
  (Iy2) edge[->] node[pos=.4,above] {$a\I_2$} (Ix2)
  (Py2) edge[->] node[pos=.4,above] {$a\P_2$} (Px2)
  (y3) edge[openmorphism] (z1)
  (Iy3) edge[->] node[pos=.35,above] {$b\I_3$} (Iz1)
  (Py3) edge[->] node[pos=.35,above] {$b\P_3$} (Pz1)
  ;

\path (Iy3.north east) edge[->,bend right=25] node[pos=.7,right] {$k$} (Iy2.south east);

% interior of the y-nodes repainted no so as to come on top
\path
  (Iy2) edge[->] node[left] {$d_{y_2}$} (Py2)
  ;
}
\colorlet{maybeWeak}{black}
\begin{center}
\begin{tikzpicture}[on grid,inner sep=1]
\mypattern
\end{tikzpicture}
\end{center}
%
Clearly, $\trip{y_2}{a_2}{x_2}$ and $\trip{y_3}{b_3}{z_1}$ form a mirror image of the diagram  \eqref{eq:composition induction premisse}, and because $\depth(\hat y_3)<\depth(\hat y_2)$ the induction hypothesis applies. This means that we can complete the diagram in either of the two following ways, where in both cases $v\desceq_\cU y_3$:
%
\colorlet{maybeWeak}{weakColor}
\begin{equation}\label{eq:composition induction proof}\mkern-6mu
\begin{tikzpicture}[on grid,inner sep=1,baseline=(Py2)]
\mypattern
\node (Ix3) [below=of Px2] {};
\node (Px3) [below=of Ix3] {};
\node (Iv) [below=of Py3] {};
\node (Pv) [below=of Iv] {};
\path
  (Ix3) edge[->] 
        node[opengraph] (x3) {} 
        node[left] {$d_{x_3}$} (Px3)
  (Iv) edge[->]
       node[opengraph] (v) {}
%       node[left] {$d_y$} 
       (Pv)
  (Ix3) edge[->] node[left] {$u_{x_3}$} (Px2)
  ;
%
\path[morphism]
  (v) edge[openmorphism] (z1)
  (Iv) edge[->] node[pos=.3,below] {$\bar c\I$} (Iz1)
  (Pv) edge[->] node[pos=.5,below] {$\bar c\P$} (Pz1)
  (x3) edge[openmorphism] (v)
  (Ix3) edge[->] node[pos=.5,above right] {$\bar d\I$} (Iv)
  (Px3) edge[->] node[pos=.3,above right] {$\bar d\P$} (Pv)
  ;

\path (Iv.north) edge[->,bend right=25] node[pos=.7,right] {$\bar s$} (Iy3.south east);

% repaint an arrow
\path[morphism]
  (Py3) edge[->] (Pz1);
\path 
  (Iv) edge[->]
%       node[opengraph] (v) {}
       node[pos=.3,left] {$d_v$} 
       (Pv);

\end{tikzpicture}
\!\!\!\!
\begin{tikzpicture}[on grid,inner sep=1,baseline=(Py2)]
\mypattern
\node (Iz2) [below=of Pz1] {};
\node (Pz2) [below=of Iz2] {};
\node (Iv) [below=of Py3] {};
\node (Pv) [below=of Iv] {};
\path
  (Iz2) edge[->] node[opengraph] (z2) {} node[right] {$d_{z_2}$} (Pz2)
  (Iv) edge[->]
       node[opengraph] (v) {}
%       node[left] {$d_y$} 
       (Pv)
  (Iz2) edge[->] node[right] {$u_{z_2}$} (Pz1)
  ;
%
\path[morphism]
  (z2) edge[openmorphism] (v)
  (Iz2) edge[->] node[pos=.4,above] {$\bar c\I$} (Iv)
  (Pz2) edge[->] node[pos=.4,above] {$\bar c\P$} (Pv)
  (v.122) edge[openmorphism] (x2)
  (Iv.west) edge[->] node[pos=.4,above=.25] {$\bar d\I$} (Ix2.south east)
  (Pv.west) edge[->] node[pos=.5,above=.3] {$\bar d\P$} (Px2.south east)
  ;

\path (Iv.north) edge[->,bend right=25] node[pos=.3,right] {$\bar s$} (Iy3.south east);

% repaint an arrow
\path[morphism]
  (Py2) edge[->] (Px2);
\path 
  (Iv) edge[->]
%       node[opengraph] (v) {}
       node[pos=.5,right] {$d_v$} 
       (Pv);
\end{tikzpicture}\mkern-6mu
\end{equation}
%
In the left diagram, besides \eqref{X1} and \eqref{X2}, the following hold: $\inlab{Y1} \bar s; b'_3 = \bar c\I$ (from \eqref{A1}) and $\inlab{Y2} \bar d\I;\bar s; u_{y_3}; a\P_2 = u_{x_3}$ (from \eqref{A2}). In the right diagram, besides  \eqref{X1} and \eqref{X2}, the following hold: $\inlab{Z1} \bar c\I;\bar s ; b\I_3; d_{z_1} = u_{z_2}$ (from \eqref{B1}) and $\inlab{Z2} \bar s; u_{y_3}; a\P_2 = d_v; \bar d\P$ (from \eqref{B2}).


In the right hand side diagram, note that also the following equality holds.
\[ \bar s;k;a\I_2;d_{x_2}=\bar s;k;d_{y_2};a\P_2\stackrel{\eqref{X1}}{=}\bar s;u_{y_3};a\P_2\stackrel{\eqref{Z2}}{=}d_v;\bar d\P = \bar d\I;d_{x_2} \enspace. \]
% \[ \bar s;k;a\I_2;d_{x_2}=\bar s;k;d_{y_2};a\P_2 = \bar s;u_{y_3};a\P_2=\bar d\I;d_{x_2} \enspace. \]
\Cref{lem:interface flexibility} therefore implies that the open graph morphism $e$ with $e\I=\bar s;k;a\I_2$ and  $e\P=\bar d\P$ satisfies $\trip {v}{e}{x_2}\in \rgen_\cX$. We may therefore, without loss of generality, choose $\bar d$ to equal this $e$; i.e., we may assume $\inlab{K}\bar d\I=\bar s;k;a\I_2$.

\medskip
The left hand side of \eqref{eq:composition induction proof} implies Clause~2 of the lemma, whereas (with the proviso in the previous paragraph) the right hand side implies Clause~1, in both cases setting $c=\bar d$, $d=\bar c$ and $s=\bar s;k$. In particular, in both cases $v\desceq_\cU y_3 \childof_\cU y_2$ and hence $v\desceq_\cU y_2$; and we need to prove the required commutation properties.
\begin{itemize}
\item For the left hand side of \eqref{eq:composition induction proof} we need to prove $\eqref{B1}~ c\I;s;a\I_2;d_{x_2} = u_{x_3}\P$ and $\eqref{B2}~  s;u_{y_2};b\P_1=d_v;d\P$ with $c=\bar d$, $d=\bar c$ and $s=\bar s;k$, that is $\bar d\I;\bar s;k;a\I_2;d_{x_2} = d_v;\bar c\P$ and $\bar s;k;u_{y_2};b\P_1=d_v;\bar c\P$. Indeed, 


\[\begin{array}{c}
  \bar d\I;\bar s;k;a\I_2;d_{x_2}
 = \bar d\I;\bar s;k;d_{y_2};a\P_2
 \stackrel{\eqref{X1}}{=} \bar d\I;\bar s;u_{y_3};a\P_2
 \stackrel{\eqref{Y2}}{=} u_{x_3} \\[\smallskipamount]
\bar s;k;u_{y_2};b\P_1
\stackrel{\eqref{X2}}{=} \bar s;d_{y_3};b\P_3
= \bar s;b\I_3; d_{z_1}
\stackrel{\eqref{Y1}}{=} \bar c\I; d_{z_1}
 = d_v;\bar c\P
\end{array}\]

% \[\begin{array}{c}
% c\I;s;a\I_2;d_{c_2}
%  = \bar d\I;\bar s;k;d_{y_2};a\P_2
%  = \bar d\I;\bar s;u_{y_3};a\P_2
%  = u_{x_3} \\[\smallskipamount]
% s;u_{y_2};b\I_1
%  = \bar s;k;u_{y_2};b\I_1
%  = \bar s;d_{y_3};b\P_3
%  = d_v;d\P
%  = d_v;\bar c\P
% \end{array}\]

\item For the right hand side of \eqref{eq:composition induction proof}, we need to prove $\eqref{A1}~s;a\I_2=c\I$ and $\eqref{A2}~d\I;s;u_{y_2};b\P_1=u_{z_2}$ with the same substitution, that is $\bar s;k;a\I_2=\bar d\I$ and $\bar c\I;\bar s;k;u_{y_2};b\P_1=u_{z_2}$. Indeed, 
\[\begin{array}{c}
\bar s;k;a\I_2
{=} \bar d\I \qquad \mbox{ by the above assumption \eqref{K}}\\[\smallskipamount]
\bar c\I;\bar s;k;u_{y_2};b\P_1
\stackrel{\eqref{X2}}{=} \bar c\I;\bar s;d_{y_3};b\P_3 
= \bar c\I;\bar s; b\I_3; d_{z_1} 
\stackrel{\eqref{Z1}}{=} u_{z_2}
\end{array}\]
\end{itemize}
\end{enumerate}
\end{proof}
%
\begin{proof}[of \Cref{prop:syntactic reflection composes}]
Given $\trip{x_1}{a_1}{y_1}\in \rgen_\cM$ and $\trip{y_1}{b_1}{z_1}\in \rgen_\cN$, we prove by induction on $\depth(a_1)+ \depth(b_1)$ that $\trip {x_1}{a_1;b_1}{z_1}\in \rgen_{\cM;\cN}$. The case of $\trip{x_1}{a_1}{y_1}\in \rgen_\cN$ and  $\trip{y_1}{b_1}{z_1}\in \rgen_\cM$ is symmetric and omitted here.

The base case is trivial, since then $\depth(a_1)=0$, implying $x_1$ has no children and hence (immediately) $\trip {x_1}{a_1;b_1}{z_1}\in \rgen_{\cM;\cN}$. Henceforth assume $\depth(a_1)+ \depth(b_1)>0$ and let $x_2\childof x_1$; we proceed by case distinction on the clauses of \Cref{def:syntactic reflection} as applied to $\trip{x_1}{a_1}{y_1}$.

\medskip\noindent\emph{Clause~\ref{xpattern}.} There is then a $\trip{y_2}{a_2}{x_2}\in \rgen_\cM$ such that $\inlab{X} (x_1,x_2)\xpattern{a_1}{a_2} (y_1,y_2)$; hence $a_2$ and $b_1$ form the diagram  \eqref{eq:composition induction premisse}. \Cref{lem:composition induction} implies that this can be extended to one of the diagrams  \eqref{eq:composition induction result}.

\begin{itemize}
\item Left hand side: here $\trip{z_2}{d}{v}\in \rgen_\cN$ and $\trip{v}{c}{x_2}\in \rgen_\cM$ such that $\depth(d)< \depth(b_1)$ and $\depth(c)< \depth(a_1)$; hence (by the induction hypothesis) $\trip{z_2}{d;c}{x_2}\in \rgen_{\cM;\cN}$. We now prove that $(x_1,x_2)\xpattern{a_1;b_1}{d;c} (z_1,z_2)$, hence we are done.
%
\[ u_{z_2}
  \stackrel{\eqref{A2}}{=} d\I;s;u_{y_2};b\P_1
  \stackrel{\eqref{X}}{=} d\I;s;a\I_2;u_{x_2};a\P_1;b\P_1
  \stackrel{\eqref{A1}}{=} (d;c)\I;u_{x_2};(a_1;b_1)\P\enspace.
\]

\item Right hand side: here $\trip{x_3}{c}{v}\in \rgen_\cM$ and $\trip{v}{d}{z_1}\in \rgen_\cN$ such that $\depth(c)< \depth(a_1)$ and $\depth(d)< \depth(b_1)$; hence (by the induction hypothesis) $\trip{x_3}{c;d}{z_1}\in \rgen_{\cM;\cN}$. We now prove that $x_1\childof x_2$ and $(x_1,x_3) \vpattern[k]{a_1;b_1}{c;d} z_1$ with $\inlab{Z} k=c\I;s;a\I_2$, hence we are done.
%
\[\begin{array}{c}
k;d_{x_2}
\stackrel{\eqref{Z}}{=} c\I;s;a\I_2;d_{x_2}
 \stackrel{\eqref{B1}}{=} u_{x_3} \\[\smallskipamount]
k;u_{x_2};(a_1;b_1)\P
\stackrel{\eqref{Z}}{=} c\I;s;a\I_2;u_{x_2};a\P_1;b\P_1
\stackrel{\eqref{X}}{=} c\I;s;u_{y_2};b\P_1
\stackrel{\eqref{B2}}{=} c\I;d_v;d\P
 = d_{x_3};(c;d)\P
\end{array}\]
\end{itemize}
%
\emph{Clause~\ref{vpattern}.} There is then a $\trip{x_3}{a_3}{y_1}\in \rgen_\cM$ for some $x_3\childof x_2$ such that $(x_1,x_3)\vpattern[k]{a_1}{a_3} y_1$. Then it is immediate to show that also $(x_1,x_3) \vpattern[k]{a_1;b_1}{a_3;b_1} z_1$, hence we are done.


\end{proof}
