\section{Model preservation}
\label{sec:preservation}

In order to reason concisely about model preservation, we call a graph morphism $f:I\to P$ \emph{$e$-prefixed}, for some graph morphism $e:I\to P'$, if $f=e;f'$ for some graph morphism $f':P'\to P$. In practice, we often use this for $e$ that are epi, in which case $f'$ is in fact uniquely defined.

To define model preservation, we have to take into account that (as observed earlier) a triple $\trip vaw$ in general cannot be expected to carry over any $\hat v$-model $g$ to some $\hat w$-model: rather, $g$ should at least be $a\I$-prefixed for this to make sense, at which point the $\hat w$-model is the `remnant'' of $g$ after $a\I$. In fact, we go one step further and only consider preservation for $u_v$-prefixed models of $\hat v$, where $u_v$ itself is required to be $a\I$-prefixed. (Such a $u_v$-prefixed model is thus certainly an $a\I$-prefixed model.) This additional complication is required to make our induction proofs work.
%
\begin{definition}[model preservation]
Let $\cM\of \cT\to\cU$ be a morphism.
\begin{itemize}[topsep=\smallskipamount]
\item $\trip vaw\in \cM$ \emph{preserves a $\hat v$-model $g$} if $g=a\I;g'$ implies $g'\sat \hat w$.

\item $\trip vaw\in \cM$ \emph{preserves models} if $a\I$ is epi, $u_v$ is $a\I$-prefixed, and $\trip vaw$ preserves all $u_v$-prefixed $\hat v$-models. 

\item $\cM$ preserves models if $\rt\cM$ preserves models.
\end{itemize}
\end{definition}
%
We will use $\cM_\pgen\subseteq \cM$ to denote the subset of triples $\trip vaw\in\cM$ such that $a\I$ is epi and $u_v$ is $a\I$-prefixed. If $\trip vaw\in \cM_\pgen$, then the remnant of $u_v$ after $a\I$ is uniquely defined, and it is useful to have a default notation for it. Henceforth, we write $a\U$ for the unique graph morphism such that $u_v=a\I;a\U$. For a visualisation see \Cref{fig:fusion-preservation}.
%
\begin{figure}
	\subcaptionbox{
   $a\I$ is epi, \\ $u_v$ is $a\I$-prefixed
   \label{fig:fusion-preservation}
}[.3\textwidth][c]{
\begin{tikzpicture}[on grid,baseline=(current bounding box.south)]
	\node (Iv) {};
	\node (Pv) [below=2 of Iv] {};
	\node (Prt) [above right=of Iv] {};
	
	\path
	(Iv) edge[->] node[doubleopengraph] {} node[left] {$d_v$} (Pv)
	(Iv) edge[->] node[pos=.3,above=.1] {$u_v$} (Prt);
	
	\node (Iw) [right=2.5 of Iv] {};
	\node (Pw) [below=2 of Iw] {};
	
	\path
	(Iw) edge[->] node[doubleopengraph] {} node[right] {$d_w$} (Pw)
    (Iw) edge[->] node[pos=.4,above=.1] {$a\U$} (Prt);
	
	\path[morphism]
	(Iv) edge[->>] node[below] {$a\I$} (Iw)
	(Pv) edge[->] node[above] {$a\P$} (Pw);
    
    \node [below=1.4 of Pw] {};
\end{tikzpicture}
}%
\subcaptionbox{
	$a\in \fuse$
    \label{fig:fusion-def}
}[.35\textwidth][c]{
\begin{tikzpicture}[on grid,baseline=(current bounding box.south)]
	\node (Iv) {};
	\node (Pv) [below=2 of Iv] {};
	\node (Prt) [above right=1 and 1.5 of Iv] {};
	
	\path
	(Iv) edge[->] node[doubleopengraph] {} node[left] {$d_v$} (Pv)
	(Iv) edge[->] node[pos=.4,above=.1] {$u_v$} (Prt);
    
	\node (Iw) [right=3 of Iv] {};
	\node (Pw) [below=2 of Iw] {};
	
	\path
	(Iw) edge[->] node[doubleopengraph] {} node[right] {$d_w$} (Pw)
    (Iw) edge[->] node[pos=.4,above=.1] {$a\U$} (Prt);
	
	\path[morphism]
	(Iv) edge[->>] node[pos=.7,below] {$a\I$} (Iw)
	(Pv) edge[->] node[above] {$a\P$} (Pw);

    \node (Qv) [text=gray,above right=1 and 1.5 of Pv,inner sep=1] {$Q_v$};
	\path[allcolor=gray]
    (Qv) edge[-{Straight Barb[gray,length=5pt,width=14pt]},white] +(-4mm,2.7mm)
    (Prt) edge[->] node[near end,left] {$d'_v$} (Qv)
    (Pv) edge[->] node[above] {$u'_v$} (Qv)
	(Pw) edge[->] node[above=.1] {$a\Q$} (Qv);
    
    \node [below=1.4 of Pw] {};
\end{tikzpicture}}%
\subcaptionbox{
	``Only if'' of \Cref{lem:fusion}
        \label{fig:fusion-proof}
}[.35\textwidth][c]{
\begin{tikzpicture}[on grid,baseline=(current bounding box.south)]
	\node (Iv) {};
	\node (Pv) [below=2 of Iv] {};
	\node (Prt) [above right=1 and 1.5 of Iv] {};
	
	\path
	(Iv) edge[->] node[doubleopengraph] {} node[left] {$d_v$} (Pv)
	(Iv) edge[->] node[pos=.4,above=.1] {$u_v$} (Prt);
    
	\node (Iw) [right=3 of Iv] {};
	\node (Pw) [below=2 of Iw] {};
	
	\path
	(Iw) edge[->] node[doubleopengraph] {} node[right] {$d_w$} (Pw)
    (Iw) edge[->] node[pos=.4,left=.1] {$a\U$} (Prt);
	
	\path[morphism]
	(Iv) edge[->>] node[pos=.7,below] {$a\I$} (Iw)
	(Pv) edge[->] node[pos=.4,above] {$a\P$} (Pw);

    \node (Qv) [text=gray,above right=1 and 1.5 of Pv,inner sep=1] {$Q_v$};
	\path[allcolor=gray]
    (Qv) edge[-{Straight Barb[gray,length=5pt,width=14pt]},white] +(-4mm,2.7mm)
    (Prt) edge[->] node[near end,left] {$d'_v$} (Qv)
    (Pv) edge[->] node[above] {$u'_v$} (Qv)
	(Pw) edge[->] node[above=.1] {$a\Q$} (Qv);
    
    \node (G) [below right=of Pw,inner sep=2] {};
    \path[allcolor=ForestGreen]
    (Prt) edge[bend left=70,->] node[above right] {$g$} (G)
    (Pv) edge[pos=.4,bend right,->] node[above] {$h$} (G.south west)
    (Qv) edge[dashed,bend right,->] node[below=.1] {$h''$} (G)
    (Pw) edge[pos=.6,dashed,->] node[above=.1] {$h'$} (G);
    
    \node [below=1.4 of Pw] {};
\end{tikzpicture}
}%

	\caption{Various preservation-related visualisations}
	\label{fig:fusion}
\end{figure}

Note that for the root triple $(v,a,w)=\rt\cM$, we have $a\I=\id_I$, which is epi and implies that $g=a\I;g'$ if and only if $g=g'$; and also $u_v=\id_I$, which is itself $a\I$-prefixed and implies that every $\hat v$-model is $u_v$-prefixed. It follows that $\rt\cM\in\cM_\pgen$ and model preservation (of $\cM$) simply means that all $\cT$-models are also $\cU$-models, as expected.
%
\begin{comment}
\begin{definition}[strong model preservation]
Let $\cM\of \cT\to\cU$ be a morphism and let $\trip vaw\in\cM$.
\begin{itemize}[topsep=\smallskipamount]
\item $a$ \emph{preserves} an $a\I$-prefixed $\hat v$-model $g=a\I;g'$ with witness $h$ if $h=a\P;h'$ for some $h'$ such that $g'\sat \hat w$ with witness $h'$.
		
\item Let $u:I_v\to P$ be an $a\I$-prefixed graph morphism. $a$ \emph{strongly preserves $u$-prefixed models} if $a$ preserves all $u$-prefixed $\hat v$-model/witness pairs.
		
\item $\cM$ strongly preserves models if $\rt\cM$ strongly preserves $\id$-prefixed models.
\end{itemize}
\end{definition}
\end{comment}
%
Like for model reflection, we prove model preservation for morphisms with a certain ``syntactic" structure. First, we introduce the concept of a \emph{fusion}, illustrated in \Cref{fig:fusion-def}.

\begin{definition}\label{def:fusion}
Let $\cM\of \cT\to\cU$ be a condition tree morphism. A triple $\trip vaw\in\cM_\pgen$ is a \emph{fusion} if the pushout of $u_v$ over $d_v$ is $a\P$-prefixed. The set of fusions is denoted $\fuse$.
\end{definition}
%
The term \emph{fusion} is chosen because, essentially, $w$ tests for the same structure as $v$, except that some of the ``parent'' pattern (i.e., the target graph of $u_v$) is \emph{fused} into the pattern $P_w$.

Fusions satisfy the following technical property:
%
\begin{lemma}\label{lem:fusion}
If $\trip vaw\in\fuse$, then $u_v;g=d_v;h$ if and only if $a\U;g=d_w;h'$ with $h=a\P;h'$.
\end{lemma}
%
This can be interpreted as follows: if we ignore the children of $v$ and $w$, then $u_v;g$ is a $\hat v$-model with witness $h$ if and only if $a\U;g$ is a $\hat w$-model with witness $h'$, where $h=a\P;h'$ expresses the relationship between the witnesses.
%
\begin{proof} Let $\trip vaw\in\fuse$, meaning that $a\I$ is epi and we have a commuting diagram as in \Cref{fig:fusion-def}.
\begin{description}[topsep=\itemsep]
\item[If.] Assume $a\U;g=d_w;h'$ and let $h=a\P;h'$; then $u_v;g=a\I;a\U;g=a\I;d_w;h'=d_v;a\P;h'=d_v;h$.
\item[Only if.] (Visualised in \Cref{fig:fusion-proof}.) Assume $u_v;g=d_v;h$; then by the pushout property, there is a unique morphism $h''$ such that $d'_v;h''=g$ and $u'_v;h''=h$. Let $h'=a\Q;h''$; then the latter equation implies $h=u'_v;h''=a\P;a\Q;h''=a\P;h'$. It follows that
\[ a\I;a\U;g=u_v;g=u_v;d'_v;h''=d_v;u'_v;h''=d_v;h=d_v;a\P;h'=a\I;d_w;h' \enspace. \]
Since $a\I$ is epi, we may conclude $a\U;g=d_w;h'$.
\end{description}
\end{proof}

\begin{definition}[preservation evidence]\label{def:preservation evidence}
Let $\cM\of \cT\to\cU$ be a condition tree morphism and let $\trip vaw\in\cM$.
\begin{itemize}[topsep=\smallskipamount]
\item \emph{(Direct preservation evidence.)} Let $y\childof w$. A triple $\trip ybx$ provides \emph{direct preservation evidence for $y$}, denoted $b\in\pedir{y,a}$, if $x\childof v$ and $b\I;u_x;a\P=u_y$.
\item \emph{(Universal preservation evidence.)} A triple $\trip ybw$ provides \emph{universal preservation evidence for $a$}, denoted $b\in\peuniv{a}$, if $y\childof x\childof v$ for some $x$ and there is a mediating morphism $k\of I_y\to I_x$ such that $k;d_x=u_y$ and $k;u_x;a\P=d_y;b\P$.
\end{itemize}
\end{definition}
%
Alternatively and more succinctly, we can define these sets as follows. Let $\trip vaw\in\cM$, $y\childof w$ and $z\childof x\childof v$:
%
\begin{align*}
\pedir{a,y} ={} & \gensetof{\trip ybx\in \cM}{x\childof v, b\I;u_x;a\P=u_y} \\
\peuniv{a,z} ={} & \gensetof{\trip zbw\in\cM}{\exists k\of I_z\to I_x\st k;d_x=u_z\wedge k;u_x;a\P=b\I;d_w}
\end{align*}
%
The following lemma captures the essential properties.
%
\begin{lemma}\label{lem:preservation evidence}
Let $\cM\of \cT\to \cU$ be a condition tree morphism, and let $\trip vaw\in \cM_\pgen$.
\begin{enumerate}[topsep=\itemsep]
\item Assume $a\in\fuse$, let $u_v;g\sat \hat v$ with witness $h$, and let $y\childof w$. If there is some $\trip xby\in \pedir{a,y}$ that preserves models, then $b\I;u_x;h\nsat \hat y$.
\item If $a\in \fuse$ and for all $y\childof w$ there is some $\trip xby\in \pedir{a,y}$ that preserves models, then $\trip vaw$ preserves models.
\item If there is some $x\childof v$ such that for all $z\childof x$ there is some $\trip zbv\in\peuniv{a,z}$ that preserves models, then $\trip vaw$ preserves models.
\end{enumerate}
\end{lemma}
%
\begin{proof}
\todo[inline]{To be done}
\end{proof}
%
Let $\pgen\subseteq \cM$ be the smallest set satisfying the following recursive equation:
%
\begin{align*}
 \pgen = {} & \gensetof{\trip vaw\in \cM_\pgen}{\forall y\childof w\st \pgen\cap \pedir{a,y}\neq\emptyset} \\
   {}\cup{} & \gensetof{\trip vaw\in \cM_\pgen}{\forall z\childof x\childof v\st \pgen\cap \peuniv{a,z}\neq\emptyset} \enspace.
\end{align*}
%
The following can now be proved by induction, analogously to \Cref{lem:reflection}.
%
\begin{lemma}[syntactic preservation implies model preservation]\label{lem:preservation}
If $\cM\of\cT\to \cU$ is a condition tree morphism, then all $a\in \pgen_\cM$ preserve models.
\end{lemma}
%
\begin{corollary}
If $\cM$ is a condition tree morphism, then $\rt\cM\in\pgen$ implies that $\cM$ preserves models.
\end{corollary}